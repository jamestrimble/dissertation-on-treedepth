\chapter{Extra \McSplit\ figures}
\label{a:appendix-extra-mcsplit}

\begin{figure}[h!]
    \centering
    \includegraphics*[width=0.45\textwidth]{14-mcsplit-i-undirected/modified-mcsplit-experiment/plots/plots/mcsplain-densities}
    \caption{In each pair of MCS plain instances, the two graphs have very similar densities. The figure
        plots density of graph $G$ against density of graph $H$ in each graph pair.}
    \label{figure:mcsplain-densities}
\end{figure}

\section{Detailed run time comparisons: unlabelled MCS instances}

The cumulative plots in \Cref{c:mcsplit-i-undirected} show that \McSplit\ and \McSplitDown\ are
faster than either the clique solver or \kDown\ for the unlabelled MCS instances.  This section
presents more detailed scatter plots of per-instance run times, with one plot for each of the five
families of MCS instances.

\Cref{figure:mcsplain-runtime-mcsplitdown-cpfc-scatters} compares CP-FC and \McSplitDown.  On each of the
families of instances, \McSplitDown\ is substantially faster than CP-FC.  Indeed, there is not a
single instance on which CP-FC is faster than \McSplitDown.

\begin{figure}[htb]
    \centering
    \subfigure[][Random graphs] {
	\centering
        \includegraphics*[width=0.3\textwidth]{14-mcsplit-i-undirected/plots-based-on-ijcai-paper/more-analysis/plots/mcsplain-runtime-mcsplitdown-cpfc-random}
        \label{figure:mcsplain-runtime-mcsplitdown-cpfc-random}
    }
    \subfigure[][Regular meshes] {
	\centering
        \includegraphics*[width=0.3\textwidth]{14-mcsplit-i-undirected/plots-based-on-ijcai-paper/more-analysis/plots/mcsplain-runtime-mcsplitdown-cpfc-regular-mesh}
        \label{figure:mcsplain-runtime-mcsplitdown-cpfc-regular-mesh}
    }
    \subfigure[][Irregular meshes] {
	\centering
        \includegraphics*[width=0.3\textwidth]{14-mcsplit-i-undirected/plots-based-on-ijcai-paper/more-analysis/plots/mcsplain-runtime-mcsplitdown-cpfc-irregular-mesh}
        \label{figure:mcsplain-runtime-mcsplitdown-cpfc-irregular mesh}
    }
    \subfigure[][Bounded valence graphs] {
	\centering
        \includegraphics*[width=0.3\textwidth]{14-mcsplit-i-undirected/plots-based-on-ijcai-paper/more-analysis/plots/mcsplain-runtime-mcsplitdown-cpfc-bv}
        \label{figure:mcsplain-runtime-mcsplitdown-cpfc-bv}
    }
    \subfigure[][Irregular BV graphs] {
	\centering
        \includegraphics*[width=0.3\textwidth]{14-mcsplit-i-undirected/plots-based-on-ijcai-paper/more-analysis/plots/mcsplain-runtime-mcsplitdown-cpfc-irregular-bv}
        \label{figure:mcsplain-runtime-mcsplitdown-cpfc-irregular-bv}
    }
    \caption{\McSplitDown\ and CP-FC run times for unlabelled MCS instances, with one plot per family of instances.}
    \label{figure:mcsplain-runtime-mcsplitdown-cpfc-scatters}
\end{figure}



\Cref{figure:mcsplain-runtime-mcsplitdown-kdown-scatters} compares \kDown\ and \McSplitDown.
Again, \McSplitDown\ is consistently the faster algorithm across all five instance families.

\begin{figure}[htb]
    \centering
    \subfigure[][Random graphs] {
	\centering
        \includegraphics*[width=0.3\textwidth]{14-mcsplit-i-undirected/plots-based-on-ijcai-paper/more-analysis/plots/mcsplain-runtime-mcsplitdown-kdown-random}
        \label{figure:mcsplain-runtime-mcsplitdown-kdown-random}
    }
    \subfigure[][Regular meshes] {
	\centering
        \includegraphics*[width=0.3\textwidth]{14-mcsplit-i-undirected/plots-based-on-ijcai-paper/more-analysis/plots/mcsplain-runtime-mcsplitdown-kdown-regular-mesh}
        \label{figure:mcsplain-runtime-mcsplitdown-kdown-regular-mesh}
    }
    \subfigure[][Irregular meshes] {
	\centering
        \includegraphics*[width=0.3\textwidth]{14-mcsplit-i-undirected/plots-based-on-ijcai-paper/more-analysis/plots/mcsplain-runtime-mcsplitdown-kdown-irregular-mesh}
        \label{figure:mcsplain-runtime-mcsplitdown-kdown-irregular mesh}
    }
    \subfigure[][Bounded valence graphs] {
	\centering
        \includegraphics*[width=0.3\textwidth]{14-mcsplit-i-undirected/plots-based-on-ijcai-paper/more-analysis/plots/mcsplain-runtime-mcsplitdown-kdown-bv}
        \label{figure:mcsplain-runtime-mcsplitdown-kdown-bv}
    }
    \subfigure[][Irregular BV graphs] {
	\centering
        \includegraphics*[width=0.3\textwidth]{14-mcsplit-i-undirected/plots-based-on-ijcai-paper/more-analysis/plots/mcsplain-runtime-mcsplitdown-kdown-irregular-bv}
        \label{figure:mcsplain-runtime-mcsplitdown-kdown-irregular-bv}
    }
    \caption{\McSplitDown\ and $k\downarrow$ run times for unlabelled MCS instances, with one plot per family of instances.}
    \label{figure:mcsplain-runtime-mcsplitdown-kdown-scatters}
\end{figure}



\Cref{figure:mcsplain-runtime-mcsplitdown-clique-scatters} compares the clique encoding and \McSplitDown.
In each instance class, \McSplitDown\ outperforms the clique algorithm overall.  However, the results are
not as clear-cut as in the CP-FC and \kDown\ comparisons.  In particular, a number of instances
in the Random family were solved more quickly by the clique solver than by \McSplitDown.  Of the 2398
instances in this family that were solved by at least one of these two solvers within the time limit,
the clique encoding was faster than \McSplitDown\ on 125 instances.  All but 11 of these 125 instances
were in the densest subset of the Random instances: graphs with each edge generated with probability $0.2$.
(Other instances in the Random family had edges generated with probabilities $0.01$, $0.05$, and $0.1$.)
This suggests that, for random instances, the advantage of \McSplitDown\ is greatest in sparse graphs.

\begin{figure}[htb]
    \centering
    \subfigure[][Random graphs] {
	\centering
        \includegraphics*[width=0.3\textwidth]{14-mcsplit-i-undirected/plots-based-on-ijcai-paper/more-analysis/plots/mcsplain-runtime-mcsplitdown-clique-random}
        \label{figure:mcsplain-runtime-mcsplitdown-clique-random}
    }
    \subfigure[][Regular meshes] {
	\centering
        \includegraphics*[width=0.3\textwidth]{14-mcsplit-i-undirected/plots-based-on-ijcai-paper/more-analysis/plots/mcsplain-runtime-mcsplitdown-clique-regular-mesh}
        \label{figure:mcsplain-runtime-mcsplitdown-clique-regular-mesh}
    }
    \subfigure[][Irregular meshes] {
	\centering
        \includegraphics*[width=0.3\textwidth]{14-mcsplit-i-undirected/plots-based-on-ijcai-paper/more-analysis/plots/mcsplain-runtime-mcsplitdown-clique-irregular-mesh}
        \label{figure:mcsplain-runtime-mcsplitdown-clique-irregular mesh}
    }
    \subfigure[][Bounded valence graphs] {
	\centering
        \includegraphics*[width=0.3\textwidth]{14-mcsplit-i-undirected/plots-based-on-ijcai-paper/more-analysis/plots/mcsplain-runtime-mcsplitdown-clique-bv}
        \label{figure:mcsplain-runtime-mcsplitdown-clique-bv}
    }
    \subfigure[][Irregular BV graphs] {
	\centering
        \includegraphics*[width=0.3\textwidth]{14-mcsplit-i-undirected/plots-based-on-ijcai-paper/more-analysis/plots/mcsplain-runtime-mcsplitdown-clique-irregular-bv}
        \label{figure:mcsplain-runtime-mcsplitdown-clique-irregular-bv}
    }
    \caption{\McSplitDown\ and clique run times for unlabelled MCS instances, with one plot per family of instances.  In the first plot,
            the densest instances ($p=0.2$) are highlighted in orange.}
    \label{figure:mcsplain-runtime-mcsplitdown-clique-scatters}
\end{figure}

\FloatBarrier

\section{Detailed run time comparisons: labelled MCS instances}

\Cref{figure:mcs33ved-runtime-mcsplitdown-cpfc-scatters} compares \McSplitDown\ and CP-FC run times
on the five families of MCS instances with labels on vertices and edges.  As in the unlabelled case,
\McSplitDown\ is the clear winner for each of the five families.

\begin{figure}[htb]
    \centering
    \subfigure[][Random graphs] {
	\centering
        \includegraphics*[width=0.3\textwidth]{14-mcsplit-i-undirected/plots-based-on-ijcai-paper/more-analysis/plots/mcs33ved-runtime-mcsplitdown-cpfc-random}
        \label{figure:mcs33ved-runtime-mcsplitdown-cpfc-random}
    }
    \subfigure[][Regular meshes] {
	\centering
        \includegraphics*[width=0.3\textwidth]{14-mcsplit-i-undirected/plots-based-on-ijcai-paper/more-analysis/plots/mcs33ved-runtime-mcsplitdown-cpfc-regular-mesh}
        \label{figure:mcs33ved-runtime-mcsplitdown-cpfc-regular-mesh}
    }
    \subfigure[][Irregular meshes] {
	\centering
        \includegraphics*[width=0.3\textwidth]{14-mcsplit-i-undirected/plots-based-on-ijcai-paper/more-analysis/plots/mcs33ved-runtime-mcsplitdown-cpfc-irregular-mesh}
        \label{figure:mcs33ved-runtime-mcsplitdown-cpfc-irregular mesh}
    }
    \subfigure[][Bounded valence graphs] {
	\centering
        \includegraphics*[width=0.3\textwidth]{14-mcsplit-i-undirected/plots-based-on-ijcai-paper/more-analysis/plots/mcs33ved-runtime-mcsplitdown-cpfc-bv}
        \label{figure:mcs33ved-runtime-mcsplitdown-cpfc-bv}
    }
    \subfigure[][Irregular BV graphs] {
	\centering
        \includegraphics*[width=0.3\textwidth]{14-mcsplit-i-undirected/plots-based-on-ijcai-paper/more-analysis/plots/mcs33ved-runtime-mcsplitdown-cpfc-irregular-bv}
        \label{figure:mcs33ved-runtime-mcsplitdown-cpfc-irregular-bv}
    }
    \caption{\McSplitDown\ and CP-FC run times for vertex and edge labelled MCS instances, with one plot per family of instances.}
    \label{figure:mcs33ved-runtime-mcsplitdown-cpfc-scatters}
\end{figure}


\Cref{figure:mcs33ved-runtime-mcsplitdown-clique-scatters} compares \McSplitDown\ and the clique
encoding on the labelled instances.  In each of the five families, the clique algorithm
is the clear winner for all but the most trivial instances.

\begin{figure}[htb]
    \centering
    \subfigure[][Random graphs] {
	\centering
        \includegraphics*[width=0.3\textwidth]{14-mcsplit-i-undirected/plots-based-on-ijcai-paper/more-analysis/plots/mcs33ved-runtime-mcsplitdown-clique-random}
        \label{figure:mcs33ved-runtime-mcsplitdown-clique-random}
    }
    \subfigure[][Regular meshes] {
	\centering
        \includegraphics*[width=0.3\textwidth]{14-mcsplit-i-undirected/plots-based-on-ijcai-paper/more-analysis/plots/mcs33ved-runtime-mcsplitdown-clique-regular-mesh}
        \label{figure:mcs33ved-runtime-mcsplitdown-clique-regular-mesh}
    }
    \subfigure[][Irregular meshes] {
	\centering
        \includegraphics*[width=0.3\textwidth]{14-mcsplit-i-undirected/plots-based-on-ijcai-paper/more-analysis/plots/mcs33ved-runtime-mcsplitdown-clique-irregular-mesh}
        \label{figure:mcs33ved-runtime-mcsplitdown-clique-irregular mesh}
    }
    \subfigure[][Bounded valence graphs] {
	\centering
        \includegraphics*[width=0.3\textwidth]{14-mcsplit-i-undirected/plots-based-on-ijcai-paper/more-analysis/plots/mcs33ved-runtime-mcsplitdown-clique-bv}
        \label{figure:mcs33ved-runtime-mcsplitdown-clique-bv}
    }
    \subfigure[][Irregular BV graphs] {
	\centering
        \includegraphics*[width=0.3\textwidth]{14-mcsplit-i-undirected/plots-based-on-ijcai-paper/more-analysis/plots/mcs33ved-runtime-mcsplitdown-clique-irregular-bv}
        \label{figure:mcs33ved-runtime-mcsplitdown-clique-irregular-bv}
    }
    \caption{\McSplitDown\ and clique run times for vertex and edge labelled MCS instances, with one plot per family of instances.}
    \label{figure:mcs33ved-runtime-mcsplitdown-clique-scatters}
\end{figure}

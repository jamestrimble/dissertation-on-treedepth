\documentclass[notitlepage]{report}
\usepackage[left=1.5in, right=1.5in, top=1.5in, bottom=1.5in]{geometry}

%\usepackage{titling}
\usepackage{lipsum}
\usepackage{algpseudocode}

%\pretitle{\begin{center}\Huge\bfseries}
%\posttitle{\par\end{center}\vskip 0.5em}
%\preauthor{\begin{center}\Large\ttfamily}
%\postauthor{\end{center}}
%\predate{\par\large\centering}
%\postdate{\par}

\newcommand{\McSplit}{\textproc{McSplit}}

\title{Partitioning Algorithms for Induced Subgraph and Supergraph Problems}
\author{James Trimble}
\begin{document}

\maketitle
\thispagestyle{empty}

\section*{Draft Abstract}
%\begin{abstract}

This dissertation introduces the \McSplit\ family of algorithms for two
    closely-related \NP-hard problems that involve finding a large induced
    subgraph contained by each of two input graphs: the induced subgraph
    isomorphism problem and the maximum common subgraph problem.

The \McSplit\ algorithms resemble forward-checking constrant programming
algorithms, but use problem-specific data structures that allow
multiple, identical domains to be stored without duplication.  These data structures
enable simple and fast constraint propagation algorithms and very fast calculation of
upper bounds.  Specialised versions of the algorithms for sparse and dense graphs are
described and implemented.
The resulting algorithms are over an order of magnitude faster
than the best existing algorithm for maximum common induced subgraph on unlabelled
graphs, and outperform the state of the art on many classes of induced subgraph
isomorphism instances.

A further advantage of the \McSplit\ data structures is that variables and
values are treated identically. This allows us to choose to branch on variables
representing vertices of either input graph with no overhead.  An extensive
set of experiments shows that such two-sided branching can be particularly
beneficial if the two input graphs have very different orders or densities.

Finally, we turn from subgraphs to supergraphs, tackling the problem of
finding a small graph that contains every member of a given family of graphs
as an induced subgraph.  Exact and heuristic techniques are developed for
this problem, in each case using a \McSplit\ algorithm as a subroutine.
These algorithms allow us to add new terms to two entries of the
On-Line Encyclopedia of Integer Sequences.
%\end{abstract}
\end{document}


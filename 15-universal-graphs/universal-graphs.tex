\chapter{Induced universal graphs}
\label{c:mcsplit-i-undirected}

%\usepackage[letterpaper, total={6in, 8in}]{geometry}
%\usepackage[x11names, svgnames, rgb]{xcolor}
%\usepackage[utf8]{inputenc}
%\usepackage{booktabs}
%\usepackage{hyperref}
%\usepackage{tikz}
%\usepackage{amsthm}
%\usetikzlibrary{snakes,arrows,shapes}
%\usepackage{amsmath}
%
%\usepackage{caption}
%\captionsetup{width=0.8\textwidth,font=footnotesize}
%


%\newtheorem{proposition}{Proposition}

\section{Introduction}

Given a collection $\calF$ of graphs, graph $G$ is \emph{induced universal} for
$\calF$ if and only if each element of $\calF$ is an induced subgraph of $G$.
Graph $G$ is a \emph{minimal} induced universal graph for $\calF$ if it has as
few vertices as possible.  This chapter covers problem of finding
small induced graphs for a family of graphs, both exactly and heuristically.
My motivation for writing this short chapter is threefold.
First, the problem generalises the minimum common supergraph problem
\cite{DBLP:journals/computing/BunkeJK00}---a natural ``companion problem''
to maximum common subgraph---from two input graphs to arbitrarily many.  Second, we
can use McSplit as a fast subgraph isomorphism solver for small graphs as part
of our algorithm; thus, this chapter provides a new application for McSplit.
Third, although much theoretical work has been done to study the size of minimum
induced universal graphs for vertex counts tending to infinity, little work has
been done to develop algorithms
for the problem, and it seems worthwhile to find terms of integer sequences for
small graphs to complement the known asymptotic results.

We write $f(k)$ to denote the order (that is, the
number of vertices) of a minimal induced universal graph for the family of all
graphs on $k$ vertices.  Moon showed that $f(k) \leq 2^{(k-1)/2}$
\cite{moon_1965}, and Alon showed that $f(k) = (1 + o(1))2^{(k-1)/2}$
\cite{alon2017asymptotically}.  There is an extensive literature on bounds on
the order of minimal induced universal subgraphs for many families of graphs;
see the references in \cite{alon2017asymptotically}.
However, to my knowledge the only
existing systematic attempt to find exact values for families of small
graphs is a Mathematics Stack Exchange answer by James Preen that presents results for families of small connected graphs obtained by brute-force search with the Maple programming language \cite{preen_math_se}.

This paper uses a brute-force approach similar to that of Preen to find minimal
induced universal graphs for families of small \emph{general} graphs.
Section~\ref{sec:method} describes our computer search method, including an
optimisation to reduce run times.  Sections~\ref{sec:results5} presents exact
values of $f(k)$ for $k \leq 5$ along with
the corresponding counts of minimal induced universal subgraphs.
Section~\ref{sec:f6} gives the value of $f(6)$ and and
Section~\ref{sec:f7} gives bounds on $f(7)$; in each of these sections, the lower
bound is proven and a graph obtained by heuristic search demonstrating the
upper bound is given.  Finally, Section~\ref{sec:trees} gives results for
families of trees.

\section{Generating all induced universal graphs}\label{sec:method}

Given $k$ and $n$,
I used a simple brute-force approach to find the set of
$n$-vertex graphs that are induced universal for the family $\calF(k)$ of all $k$-vertex
graphs.  Each possible $n$-vertex graph $G$ from Brendan McKay's lists of
graphs\footnote{\url{https://users.cecs.anu.edu.au/~bdm/data/graphs.html}} was
tested in turn.  I used a Python implementation of the McSplit
algorithm \cite{DBLP:conf/ijcai/McCreeshPT17} to test whether each $k$-vertex
graph is an induced subgraph of $G$.  Although the McSplit algorithm was designed for
the more general maximum common induced subgraph problem, it performs well on
the subgraph isomorphism problem if both input graphs are small.  (Although
I have described the application of this method to families of all graphs of a given order,
the method could easily be applied to any family $\calF$ of graphs, and indeed
Section~\ref{sec:trees} gives results from the application of this method
to families of trees.)

For those graphs $G$ that do not contain a copy of each element of $\calF(k)$,
it improves the speed of our algorithm if we can iterate over
$\calF(k)$ in an order that allows us to quickly reject $G$ by finding a graph
that is not an induced subgraph of $G$.  The approach I used was to sort
$\calF(k)$ in descending order of number of isomorphisms.  In particular, this
means that the clique $K_k$ and the independent set $I_k$ are the first two
graphs tested.

The full set of experiments described in this paper, run sequentially,
took less than an hour to complete on a laptop with an Intel Core i5-6200U CPU
and 8 GB RAM.\footnote{The code and
results from this paper, including lists of minimal induced universal graphs
in graph6 format, are available from
\url{https://github.com/jamestrimble/small-universal-graphs}}

\section{Results for \texorpdfstring{$k \leq 5$}{k<=5}}\label{sec:results5}

Table~\ref{tab:graphresults} shows, for $0 \leq k \leq 5$, the value of $f(k)$.
The table also shows $F(k)$, the number of minimal induced universal
graphs for the family of all $k$-vertex graphs.  To my knowledge, the value of
$f(5)$ and the values of $F(k)$ have not been published previously.

The values $f(1)$, $f(2)$ and $f(3)$ are equal to to the simple lower bound $2k
- 1$ given in a question by ``Chain Markov'' on Mathematics Stack Exchange
  \cite{math_se_question}.  The values $f(4)$ and $f(5)$ are equal to a lower
  bound given in a comment by ``bof'' on the same question: $f(k) \geq 2k$ if $k
  \geq 4$.  (For $k < 10$, this bound improves upon Moon's lower bound $f(k)
  \leq 2^{(k-1)/2}$.) To briefly summarise the proof, if $f(k) \leq 2k$ then $G$
  must be a split graph (that is, a graph whose vertices can be partitioned
  into a clique and an independent set); therefore, $G$ cannot contain the
  cycle $C_4$ as an induced subgraph.  An example of an 8-vertex induced
  universal graph for the family of 4-vertex graphs was given by ``Chain
  Markov'' as a comment on the same question.

\begin{table}[h!]
\centering
\begin{tabular}{r r r}
 \toprule
 $k$ & $f(k)$ & $F(k)$ \\ [0.5ex]
 \midrule
 0 & 0 & 1 \\
 1 & 1 & 1 \\
 2 & 3 & 2 \\
 3 & 5 & 5 \\
 4 & 8 & 438 \\
 5 & 10 & 22 \\
 \bottomrule
\end{tabular}
\caption{For each $k$, $f(k)$ is the minimum order of a graph containing all $k$-vertex graphs as
induced subgraphs, and $F(k)$ is the number of distinct $f(k)$-vertex graphs that contain
all $k$-vertex graphs as induced subgraphs.}
\label{tab:graphresults}
\end{table}
%
%

Figure~\ref{fig:graphs} shows examples of minimal induced universal graphs
for the families of all graphs with three, four and five vertices
respectively.

\begin{figure}[htb]
    \centering

\begin{tikzpicture}[>=latex',line join=bevel,scale=.4]
  \pgfsetlinewidth{.5bp}
%%
\pgfsetcolor{black}
  % Edge: 3 -- 0
  \draw [] (51.166bp,32.248bp) .. controls (44.492bp,40.742bp) and (36.044bp,51.496bp)  .. (29.344bp,60.024bp);
  % Edge: 4 -- 0
  \draw [] (71.737bp,82.653bp) .. controls (60.758bp,80.98bp) and (46.814bp,78.855bp)  .. (35.882bp,77.189bp);
  % Edge: 4 -- 1
  \draw [] (104.4bp,96.268bp) .. controls (113.37bp,102.97bp) and (124.85bp,111.55bp)  .. (133.86bp,118.28bp);
  % Edge: 4 -- 3
  \draw [] (82.931bp,68.405bp) .. controls (78.728bp,58.107bp) and (73.39bp,45.028bp)  .. (69.205bp,34.773bp);
  % Node: 0
\begin{scope}
  \definecolor{strokecol}{rgb}{0.0,0.0,0.0};
  \pgfsetstrokecolor{strokecol}
  \draw (18.0bp,74.46bp) ellipse (18.0bp and 18.0bp);
\end{scope}
  % Node: 1
\begin{scope}
  \definecolor{strokecol}{rgb}{0.0,0.0,0.0};
  \pgfsetstrokecolor{strokecol}
  \draw (148.63bp,129.31bp) ellipse (18.0bp and 18.0bp);
\end{scope}
  % Node: 2
\begin{scope}
  \definecolor{strokecol}{rgb}{0.0,0.0,0.0};
  \pgfsetstrokecolor{strokecol}
  \draw (148.0bp,57.0bp) ellipse (18.0bp and 18.0bp);
\end{scope}
  % Node: 3
\begin{scope}
  \definecolor{strokecol}{rgb}{0.0,0.0,0.0};
  \pgfsetstrokecolor{strokecol}
  \draw (62.36bp,18.0bp) ellipse (18.0bp and 18.0bp);
\end{scope}
  % Node: 4
\begin{scope}
  \definecolor{strokecol}{rgb}{0.0,0.0,0.0};
  \pgfsetstrokecolor{strokecol}
  \draw (89.87bp,85.42bp) ellipse (18.0bp and 18.0bp);
\end{scope}
%
\end{tikzpicture}
\qquad
\qquad
\begin{tikzpicture}[>=latex',line join=bevel,scale=.4]
  \pgfsetlinewidth{.5bp}
%%
\pgfsetcolor{black}
  % Edge: 4 -- 0
  \draw [] (44.956bp,147.01bp) .. controls (40.082bp,141.1bp) and (34.42bp,134.24bp)  .. (29.546bp,128.33bp);
  % Edge: 5 -- 1
  \draw [] (166.49bp,80.229bp) .. controls (168.44bp,90.759bp) and (170.89bp,103.99bp)  .. (172.84bp,114.47bp);
  % Edge: 6 -- 0
  \draw [] (81.617bp,113.74bp) .. controls (68.067bp,113.87bp) and (49.632bp,114.04bp)  .. (36.095bp,114.17bp);
  % Edge: 6 -- 1
  \draw [] (117.47bp,117.89bp) .. controls (129.81bp,120.9bp) and (146.19bp,124.9bp)  .. (158.51bp,127.9bp);
  % Edge: 6 -- 2
  \draw [] (96.884bp,95.599bp) .. controls (94.148bp,78.511bp) and (90.056bp,52.957bp)  .. (87.325bp,35.896bp);
  % Edge: 6 -- 4
  \draw [] (87.484bp,127.04bp) .. controls (81.59bp,133.51bp) and (74.549bp,141.23bp)  .. (68.67bp,147.68bp);
  % Edge: 7 -- 0
  \draw [] (56.422bp,85.611bp) .. controls (48.962bp,91.188bp) and (39.886bp,97.973bp)  .. (32.429bp,103.55bp);
  % Edge: 7 -- 2
  \draw [] (75.094bp,57.147bp) .. controls (76.715bp,50.372bp) and (78.564bp,42.641bp)  .. (80.189bp,35.849bp);
  % Edge: 7 -- 4
  \draw [] (67.894bp,92.673bp) .. controls (65.43bp,107.47bp) and (61.945bp,128.39bp)  .. (59.482bp,143.18bp);
  % Edge: 7 -- 5
  \draw [] (89.1bp,72.32bp) .. controls (105.38bp,70.096bp) and (129.09bp,66.857bp)  .. (145.27bp,64.648bp);
  % Edge: 7 -- 6
  \draw [] (81.628bp,89.246bp) .. controls (84.002bp,92.431bp) and (86.52bp,95.808bp)  .. (88.898bp,99.0bp);
  % Node: 0
\begin{scope}
  \definecolor{strokecol}{rgb}{0.0,0.0,0.0};
  \pgfsetstrokecolor{strokecol}
  \draw (18.0bp,114.33bp) ellipse (18.0bp and 18.0bp);
\end{scope}
  % Node: 1
\begin{scope}
  \definecolor{strokecol}{rgb}{0.0,0.0,0.0};
  \pgfsetstrokecolor{strokecol}
  \draw (176.12bp,132.19bp) ellipse (18.0bp and 18.0bp);
\end{scope}
  % Node: 2
\begin{scope}
  \definecolor{strokecol}{rgb}{0.0,0.0,0.0};
  \pgfsetstrokecolor{strokecol}
  \draw (84.46bp,18.0bp) ellipse (18.0bp and 18.0bp);
\end{scope}
  % Node: 3
\begin{scope}
  \definecolor{strokecol}{rgb}{0.0,0.0,0.0};
  \pgfsetstrokecolor{strokecol}
  \draw (210.0bp,194.0bp) ellipse (18.0bp and 18.0bp);
\end{scope}
  % Node: 4
\begin{scope}
  \definecolor{strokecol}{rgb}{0.0,0.0,0.0};
  \pgfsetstrokecolor{strokecol}
  \draw (56.51bp,161.02bp) ellipse (18.0bp and 18.0bp);
\end{scope}
  % Node: 5
\begin{scope}
  \definecolor{strokecol}{rgb}{0.0,0.0,0.0};
  \pgfsetstrokecolor{strokecol}
  \draw (163.15bp,62.21bp) ellipse (18.0bp and 18.0bp);
\end{scope}
  % Node: 6
\begin{scope}
  \definecolor{strokecol}{rgb}{0.0,0.0,0.0};
  \pgfsetstrokecolor{strokecol}
  \draw (99.76bp,113.58bp) ellipse (18.0bp and 18.0bp);
\end{scope}
  % Node: 7
\begin{scope}
  \definecolor{strokecol}{rgb}{0.0,0.0,0.0};
  \pgfsetstrokecolor{strokecol}
  \draw (70.87bp,74.81bp) ellipse (18.0bp and 18.0bp);
\end{scope}
%
\end{tikzpicture}
\qquad
\qquad
\begin{tikzpicture}[>=latex',line join=bevel,scale=.4]
  \pgfsetlinewidth{.5bp}
%%
\pgfsetcolor{black}
  % Edge: 5 -- 0
  \draw [] (50.975bp,158.84bp) .. controls (45.582bp,156.23bp) and (39.597bp,153.34bp)  .. (34.222bp,150.74bp);
  % Edge: 5 -- 1
  \draw [] (84.643bp,161.37bp) .. controls (96.964bp,157.48bp) and (113.5bp,152.27bp)  .. (125.86bp,148.36bp);
  % Edge: 6 -- 0
  \draw [] (21.443bp,105.03bp) .. controls (20.863bp,111.4bp) and (20.211bp,118.58bp)  .. (19.631bp,124.95bp);
  % Edge: 6 -- 2
  \draw [] (31.856bp,71.036bp) .. controls (37.972bp,59.88bp) and (46.093bp,45.065bp)  .. (52.198bp,33.928bp);
  % Edge: 6 -- 5
  \draw [] (31.841bp,102.81bp) .. controls (39.568bp,116.7bp) and (50.773bp,136.84bp)  .. (58.539bp,150.8bp);
  % Edge: 7 -- 1
  \draw [] (135.92bp,66.785bp) .. controls (137.53bp,83.477bp) and (139.91bp,108.08bp)  .. (141.51bp,124.75bp);
  % Edge: 7 -- 2
  \draw [] (117.55bp,41.664bp) .. controls (105.65bp,36.69bp) and (89.681bp,30.016bp)  .. (77.737bp,25.024bp);
  % Edge: 7 -- 3
  \draw [] (146.34bp,62.563bp) .. controls (149.79bp,66.518bp) and (153.54bp,70.819bp)  .. (156.97bp,74.761bp);
  % Edge: 8 -- 0
  \draw [] (79.976bp,126.94bp) .. controls (66.776bp,130.34bp) and (48.816bp,134.96bp)  .. (35.628bp,138.36bp);
  % Edge: 8 -- 1
  \draw [] (114.12bp,129.78bp) .. controls (118.2bp,131.62bp) and (122.58bp,133.58bp)  .. (126.67bp,135.42bp);
  % Edge: 8 -- 3
  \draw [] (114.19bp,114.53bp) .. controls (125.71bp,109.05bp) and (141.01bp,101.78bp)  .. (152.51bp,96.32bp);
  % Edge: 8 -- 5
  \draw [] (87.424bp,137.42bp) .. controls (84.273bp,142.05bp) and (80.806bp,147.15bp)  .. (77.657bp,151.78bp);
  % Edge: 8 -- 6
  \draw [] (81.103bp,114.54bp) .. controls (68.744bp,108.69bp) and (51.929bp,100.72bp)  .. (39.582bp,94.871bp);
  % Edge: 8 -- 7
  \draw [] (105.76bp,106.01bp) .. controls (111.81bp,93.786bp) and (120.04bp,77.152bp)  .. (126.09bp,64.937bp);
  % Edge: 9 -- 0
  \draw [] (66.194bp,101.08bp) .. controls (55.929bp,109.98bp) and (41.964bp,122.1bp)  .. (31.708bp,131.0bp);
  % Edge: 9 -- 1
  \draw [] (93.68bp,100.8bp) .. controls (104.16bp,109.69bp) and (118.53bp,121.89bp)  .. (129.11bp,130.87bp);
  % Edge: 9 -- 2
  \draw [] (75.241bp,71.56bp) .. controls (72.339bp,60.696bp) and (68.624bp,46.796bp)  .. (65.707bp,35.879bp);
  % Edge: 9 -- 3
  \draw [] (97.955bp,89.017bp) .. controls (113.3bp,88.906bp) and (135.24bp,88.748bp)  .. (150.67bp,88.636bp);
  % Edge: 9 -- 5
  \draw [] (77.041bp,107.16bp) .. controls (75.021bp,119.7bp) and (72.338bp,136.36bp)  .. (70.321bp,148.89bp);
  % Edge: 9 -- 6
  \draw [] (61.949bp,88.484bp) .. controls (55.311bp,88.239bp) and (47.792bp,87.962bp)  .. (41.147bp,87.717bp);
  % Edge: 9 -- 8
  \draw [] (88.59bp,105.38bp) .. controls (88.73bp,105.64bp) and (88.871bp,105.9bp)  .. (89.011bp,106.17bp);
  % Node: 0
\begin{scope}
  \definecolor{strokecol}{rgb}{0.0,0.0,0.0};
  \pgfsetstrokecolor{strokecol}
  \draw (18.0bp,142.9bp) ellipse (18.0bp and 18.0bp);
\end{scope}
  % Node: 1
\begin{scope}
  \definecolor{strokecol}{rgb}{0.0,0.0,0.0};
  \pgfsetstrokecolor{strokecol}
  \draw (143.26bp,142.87bp) ellipse (18.0bp and 18.0bp);
\end{scope}
  % Node: 2
\begin{scope}
  \definecolor{strokecol}{rgb}{0.0,0.0,0.0};
  \pgfsetstrokecolor{strokecol}
  \draw (60.93bp,18.0bp) ellipse (18.0bp and 18.0bp);
\end{scope}
  % Node: 3
\begin{scope}
  \definecolor{strokecol}{rgb}{0.0,0.0,0.0};
  \pgfsetstrokecolor{strokecol}
  \draw (168.96bp,88.5bp) ellipse (18.0bp and 18.0bp);
\end{scope}
  % Node: 4
\begin{scope}
  \definecolor{strokecol}{rgb}{0.0,0.0,0.0};
  \pgfsetstrokecolor{strokecol}
  \draw (210.0bp,18.0bp) ellipse (18.0bp and 18.0bp);
\end{scope}
  % Node: 5
\begin{scope}
  \definecolor{strokecol}{rgb}{0.0,0.0,0.0};
  \pgfsetstrokecolor{strokecol}
  \draw (67.44bp,166.8bp) ellipse (18.0bp and 18.0bp);
\end{scope}
  % Node: 6
\begin{scope}
  \definecolor{strokecol}{rgb}{0.0,0.0,0.0};
  \pgfsetstrokecolor{strokecol}
  \draw (23.08bp,87.05bp) ellipse (18.0bp and 18.0bp);
\end{scope}
  % Node: 7
\begin{scope}
  \definecolor{strokecol}{rgb}{0.0,0.0,0.0};
  \pgfsetstrokecolor{strokecol}
  \draw (134.17bp,48.61bp) ellipse (18.0bp and 18.0bp);
\end{scope}
  % Node: 8
\begin{scope}
  \definecolor{strokecol}{rgb}{0.0,0.0,0.0};
  \pgfsetstrokecolor{strokecol}
  \draw (97.65bp,122.39bp) ellipse (18.0bp and 18.0bp);
\end{scope}
  % Node: 9
\begin{scope}
  \definecolor{strokecol}{rgb}{0.0,0.0,0.0};
  \pgfsetstrokecolor{strokecol}
  \draw (79.94bp,89.15bp) ellipse (18.0bp and 18.0bp);
\end{scope}
%
\end{tikzpicture}
\caption{Minimal induced universal graphs for the families of all
graphs with 3, 4, and 5 vertices}
\label{fig:graphs}
\end{figure}

\section{\texorpdfstring{$f(6) = 14$}{f(6)=14}}\label{sec:f6}

This section shows that $f(6) = 14$.  We begin with the lower bound.
For $k \geq 6$, we can increase by 2 the lower bound by ``bof''.  In particular,
this means that $f(6) \geq 14$.

\begin{proposition}\label{f6proposition}
    $f(k) \geq 2k + 2$ for all $k \geq 6$.
\end{proposition}
\begin{proof}

    Suppose that $G$ is an induced universal graph for the family of all graphs
    on $k$ vertices, and that $G$ has no more than $2k + 1$ vertices.  Graph
    $G$ must have $K_k$ and $I_k$ as induced sugraphs.  This clique and
    independent set may overlap by no more than one vertex, so their union must
    contain at least $2k - 1$ vertices.  Therefore it is possible to partition
    the vertices of $G$ into three sets: a clique $S_1$, an independent set
    $S_2$, and a third set $S_3$ containing at most 2 vertices.

    We will show that $G$ cannot contain as induced subgraphs both
    of the graphs in Figure~\ref{fig:boundproof}.  These graphs are complements
    of each other, and we refer to them as $H$ and $H'$ respectively.

\begin{figure}[htb]
    \centering

% \begin{tikzpicture}[>=latex',line join=bevel]
%   \tikzstyle{every node}=[draw, circle, inner sep=1pt, minimum size=.5cm]
%   \pgfsetlinewidth{.5bp}
%   \node at (0,2.9) (1) {};
%   \node at (0,2) (2) {};
%   \node at (0,1.1) (3) {};
%   \node at (1.1,2.9) (4) {};
%   \node at (1.1,2) (5) {};
%   \node at (1.1,1.1) (6) {};
%   \draw (1) -- (2) -- (3);
%   \draw (3) to [out=140,in=220] (1);
%   \draw (4) -- (5) -- (6);
%   \draw (6) to [out=40,in=-40] (4);
%   \draw (1) -- (4);
%   \draw (1) -- (5);
%   \draw (1) -- (6);
%   \draw (2) -- (4);
%   \draw (2) -- (5);
%   \draw (2) -- (6);
%   \draw (3) -- (4);
%   \draw (3) -- (5);
%   \draw (3) -- (6);
% \end{tikzpicture}
% \qquad \qquad
% \begin{tikzpicture}
%   \tikzstyle{every node}=[draw, circle, inner sep=1pt, minimum size=.5cm]
%   \pgfsetlinewidth{.5bp}
%   \node at (0,2.9) (1) {};
%   \node at (0,2) (2) {};
%   \node at (0,1.1) (3) {};
%   \node at (1.1,2.9) (4) {};
%   \node at (1.1,2) (5) {};
%   \node at (1.1,1.1) (6) {};
% \end{tikzpicture}
% \qquad \qquad
\begin{tikzpicture}[>=latex',line join=bevel]
  \tikzstyle{every node}=[draw, circle, inner sep=1pt, minimum size=.5cm]
  \pgfsetlinewidth{.5bp}
  \node at (0,2.9) (1) {};
  \node at (0,2) (2) {};
  \node at (0,1.1) (3) {};
  \node at (1.1,2.9) (4) {};
  \node at (1.1,2) (5) {};
  \node at (1.1,1.1) (6) {};
  \draw (1) -- (2) -- (3);
  \draw (3) to [out=140,in=220] (1);
  \draw (4) -- (5) -- (6);
  \draw (6) to [out=140,in=220] (4);
\end{tikzpicture}
\qquad \qquad
\begin{tikzpicture}[>=latex',line join=bevel]
  \tikzstyle{every node}=[draw, circle, inner sep=1pt, minimum size=.5cm]
  \pgfsetlinewidth{.5bp}
  \node at (0,2.9) (1) {};
  \node at (0,2) (2) {};
  \node at (0,1.1) (3) {};
  \node at (1.1,2.9) (4) {};
  \node at (1.1,2) (5) {};
  \node at (1.1,1.1) (6) {};
  \draw (1) -- (4);
  \draw (1) -- (5);
  \draw (1) -- (6);
  \draw (2) -- (4);
  \draw (2) -- (5);
  \draw (2) -- (6);
  \draw (3) -- (4);
  \draw (3) -- (5);
  \draw (3) -- (6);
\end{tikzpicture}
%%
\caption{If $k > 5$ and $G$ is a graph on $2k + 1$ or fewer vertices, then
$G$ cannot have as induced subgraphs all four of the following graphs: the
clique $K_k$, the independent set $I_k$, and
the two graphs shown. See the the proof of Proposition~\ref{f6proposition}}.
\label{fig:boundproof}
\end{figure}

    Let $G_1$ be an induced
    subgraph of $G$ that is isomorphic to $H$.  Since there are
    no edges between the two three-vertex cliques in $G_1$, it must be the case that the
    vertex set of at least one of these cliques does not intersect $S_1$.
    Since $S_2$ is an independent set in $G$, this clique must have exactly
    one vertex in $S_2$ and two vertices in $S_3$.  We can deduce, then, that
    $S_3$ contains exactly two vertices, and that these vertices are adjacent in $G$.

    Now consider graph $H'$.  Since $H'$ is an induced subgraph of $G$, it
    follows by taking complements of $H'$ and $G$ that $H$ is an induced
    subgraph of $G'$ (the complement of $G$).  We can repeat the argument
    of the previous paragraph with the roles of $S_1$ and $S_2$ reversed to
    show that the two vertices in $S_3$ must be adjacent in the complement of
    $G$, and therefore must not be adjacent in $G$.  Since we previously showed that
    these vertices are adjacent in $G$, we have a contradiction.
\end{proof}

To give an upper bound of 14 on $f(6)$, Figure~\ref{fig:adjmat14} shows the
adjacency matrix of a 14-vertex graph that is induced universal for the family
of all graphs on six vertices.  This was generated using a simple local search
algorithm.  We begin by generating a random graph on 14 vertices as follows.
We number the vertices from zero; the generator makes the
$k$ vertices numbered $0$ to $k-1$ a clique, and the $k$ vertices numbered $k-1$ to $2k-2$ an
independent set.\footnote{The clique and the independent set thus have one vertex
in common.  The proof of Proposition~\ref{f6proposition} can be modified straightforwardly
to show that there is no 14-vertex induced universal graph for this family of graphs
that contains a 6-vertex clique and a 6-vertex independent set as
induced subgraphs with disjoint vertex sets.}
Each possible edge that is not involved in either the clique
or the independent set is added with probability $1/2$.
We then repeatedly ``flip'' the status of a random edge from present to absent
or vice versa, but always leave the large clique and independent set intact.
After each flip, we count the number of 156 graphs on 6 vertices that are isomorphic
to a subgraph of our 14-vertex graph.  If the most recent flip decreased
this number, we revert it.  After each 1000 flips, we restart the algorithm
with a new random graph.

\begin{figure}[htb]
\centering
\small
\verb|0 1 1 1 1 1 0 1 1 0 0 0 1 0| \\
\verb|1 0 1 1 1 1 1 1 1 1 1 0 0 1| \\
\verb|1 1 0 1 1 1 0 1 1 0 1 0 1 0| \\
\verb|1 1 1 0 1 1 1 0 0 0 1 1 0 0| \\
\verb|1 1 1 1 0 1 0 1 0 0 0 0 0 0| \\
\verb|1 1 1 1 1 0 0 0 0 0 0 1 0 1| \\
\verb|0 1 0 1 0 0 0 0 0 0 0 1 1 1| \\
\verb|1 1 1 0 1 0 0 0 0 0 0 1 1 0| \\
\verb|1 1 1 0 0 0 0 0 0 0 0 0 1 0| \\
\verb|0 1 0 0 0 0 0 0 0 0 0 1 0 1| \\
\verb|0 1 1 1 0 0 0 0 0 0 0 1 1 0| \\
\verb|0 0 0 1 0 1 1 1 0 1 1 0 0 1| \\
\verb|1 0 1 0 0 0 1 1 1 0 1 0 0 0| \\
\verb|0 1 0 0 0 1 1 0 0 1 0 1 0 0|
\caption{The adjacency matrix of a 14-vertex induced universal graph for the family of all
six-vertex graphs}
\label{fig:adjmat14}
\end{figure}

\section{Bounds on \texorpdfstring{$f(7)$}{f(7)}}\label{sec:f7}

By Proposition~\ref{f6proposition}, we have $f(7) \geq 16$.  Figure~\ref{fig:adjmat18}
shows the adjacency matrix of an 18-vertex induced universal
graph for the family of all seven-vertex graphs. This was generated with 
the heuristic described in Section~\ref{sec:f6}, with two modifications.
First, 10000 rather than 1000
edge-flips were permitted before each restart, as this was found to be more effective
in a preliminary run of the experiment.  Second, the overlapping six-vertex clique
and independent set were replaced with a clique on seven vertices and an independent
set on seven vertices.  Again, these had one vertex in common.  (We also tried making
the clique and independent set vertex-disjoint, but did not find an 18-vertex solution
in four hours with this approach.)

Thus we have $16 \leq f(7) \leq 18$.

\begin{figure}[htb]
\centering
\small
\verb|0 1 1 1 1 1 1 1 1 1 0 1 1 1 1 1 0 0| \\
\verb|1 0 1 1 1 1 1 0 1 1 1 0 1 1 0 1 0 0| \\
\verb|1 1 0 1 1 1 1 1 0 1 0 1 0 0 0 0 0 1| \\
\verb|1 1 1 0 1 1 1 0 1 0 0 1 0 0 0 0 0 1| \\
\verb|1 1 1 1 0 1 1 1 1 0 0 0 0 0 1 0 0 0| \\
\verb|1 1 1 1 1 0 1 0 1 0 0 0 0 1 1 1 1 1| \\
\verb|1 1 1 1 1 1 0 0 0 0 0 0 0 1 0 0 0 1| \\
\verb|1 0 1 0 1 0 0 0 0 0 0 0 0 0 1 1 1 0| \\
\verb|1 1 0 1 1 1 0 0 0 0 0 0 0 1 1 1 0 1| \\
\verb|1 1 1 0 0 0 0 0 0 0 0 0 0 0 1 1 1 1| \\
\verb|0 1 0 0 0 0 0 0 0 0 0 0 0 1 0 0 1 0| \\
\verb|1 0 1 1 0 0 0 0 0 0 0 0 0 0 1 1 1 0| \\
\verb|1 1 0 0 0 0 0 0 0 0 0 0 0 0 0 1 1 1| \\
\verb|1 1 0 0 0 1 1 0 1 0 1 0 0 0 0 0 0 1| \\
\verb|1 0 0 0 1 1 0 1 1 1 0 1 0 0 0 0 0 1| \\
\verb|1 1 0 0 0 1 0 1 1 1 0 1 1 0 0 0 1 1| \\
\verb|0 0 0 0 0 1 0 1 0 1 1 1 1 0 0 1 0 1| \\
\verb|0 0 1 1 0 1 1 0 1 1 0 0 1 1 1 1 1 0|
\caption{The adjacency matrix of an 18-vertex induced universal graph for the family of all
seven-vertex graphs}
\label{fig:adjmat18}
\end{figure}

\section{Trees}\label{sec:trees}

Table~\ref{tab:treeresults} gives the order $t(k)$ of a minimal induced universal graph for
the family of $k$-vertex trees, and the number $T(k)$ of such graphs.  Figure~\ref{fig:trees}
shows one of the 66 minimal induced universal graphs for the family of 6-vertex trees.

%\section{Acknowledgements}
%
%I would like to thank Brendan McKay for helpful feedback on this paper, and Persi
%Diaconis for introducing me to induced universal graphs and for interesting email
%discussions on related topics.

\begin{table}[h!]
\centering
\begin{tabular}{r r r}
 \toprule
 $k$ & $t(k)$ & $T(k)$ \\ [0.5ex]
 \midrule
 1 & 1 & 1 \\
 2 & 2 & 1 \\
 3 & 3 & 1 \\
 4 & 5 & 2 \\
 5 & 7 & 18 \\
 6 & 9 & 66 \\
 \bottomrule
\end{tabular}
\caption{For each $k$, $t(k)$ is the minimum order of a graph containing all $k$-vertex trees as
induced subgraphs, and $T(k)$ is the number of distinct $t(k)$-vertex graphs that contain
all $k$-vertex trees as induced subgraphs.}
\label{tab:treeresults}
\end{table}

\begin{figure}[htb]
    \centering
\begin{tikzpicture}[>=latex',line join=bevel,scale=.4]
  \pgfsetlinewidth{.5bp}
%%
\pgfsetcolor{black}
  % Edge: 6 -- 0
  \draw [] (33.468bp,122.33bp) .. controls (44.467bp,128.89bp) and (59.197bp,137.66bp)  .. (70.288bp,144.27bp);
  % Edge: 6 -- 1
  \draw [] (30.834bp,100.1bp) .. controls (39.774bp,91.035bp) and (51.646bp,78.993bp)  .. (60.571bp,69.941bp);
  % Edge: 7 -- 0
  \draw [] (108.35bp,211.0bp) .. controls (103.69bp,198.96bp) and (97.434bp,182.81bp)  .. (92.754bp,170.73bp);
  % Edge: 7 -- 2
  \draw [] (128.78bp,216.26bp) .. controls (138.81bp,207.93bp) and (152.29bp,196.74bp)  .. (162.31bp,188.42bp);
  % Edge: 7 -- 3
  \draw [] (111.85bp,245.92bp) .. controls (109.9bp,257.68bp) and (107.35bp,272.94bp)  .. (105.4bp,284.67bp);
  % Edge: 8 -- 0
  \draw [] (134.88bp,109.4bp) .. controls (124.44bp,118.9bp) and (110.07bp,131.98bp)  .. (99.64bp,141.47bp);
  % Edge: 8 -- 1
  \draw [] (132.46bp,88.608bp) .. controls (119.73bp,81.799bp) and (101.96bp,72.297bp)  .. (89.227bp,65.491bp);
  % Edge: 8 -- 2
  \draw [] (154.41bp,114.43bp) .. controls (159.1bp,127.91bp) and (165.58bp,146.51bp)  .. (170.24bp,159.91bp);
  % Edge: 8 -- 4
  \draw [] (166.38bp,93.965bp) .. controls (179.82bp,91.611bp) and (198.11bp,88.409bp)  .. (211.53bp,86.058bp);
  % Edge: 8 -- 5
  \draw [] (152.57bp,79.558bp) .. controls (155.7bp,66.447bp) and (159.96bp,48.609bp)  .. (163.09bp,35.509bp);
  % Node: 0
\begin{scope}
  \definecolor{strokecol}{rgb}{0.0,0.0,0.0};
  \pgfsetstrokecolor{strokecol}
  \draw (86.17bp,153.73bp) ellipse (18.0bp and 18.0bp);
\end{scope}
  % Node: 1
\begin{scope}
  \definecolor{strokecol}{rgb}{0.0,0.0,0.0};
  \pgfsetstrokecolor{strokecol}
  \draw (73.34bp,56.99bp) ellipse (18.0bp and 18.0bp);
\end{scope}
  % Node: 2
\begin{scope}
  \definecolor{strokecol}{rgb}{0.0,0.0,0.0};
  \pgfsetstrokecolor{strokecol}
  \draw (176.16bp,176.91bp) ellipse (18.0bp and 18.0bp);
\end{scope}
  % Node: 3
\begin{scope}
  \definecolor{strokecol}{rgb}{0.0,0.0,0.0};
  \pgfsetstrokecolor{strokecol}
  \draw (102.42bp,302.6bp) ellipse (18.0bp and 18.0bp);
\end{scope}
  % Node: 4
\begin{scope}
  \definecolor{strokecol}{rgb}{0.0,0.0,0.0};
  \pgfsetstrokecolor{strokecol}
  \draw (229.48bp,82.92bp) ellipse (18.0bp and 18.0bp);
\end{scope}
  % Node: 5
\begin{scope}
  \definecolor{strokecol}{rgb}{0.0,0.0,0.0};
  \pgfsetstrokecolor{strokecol}
  \draw (167.27bp,18.0bp) ellipse (18.0bp and 18.0bp);
\end{scope}
  % Node: 6
\begin{scope}
  \definecolor{strokecol}{rgb}{0.0,0.0,0.0};
  \pgfsetstrokecolor{strokecol}
  \draw (18.0bp,113.12bp) ellipse (18.0bp and 18.0bp);
\end{scope}
  % Node: 7
\begin{scope}
  \definecolor{strokecol}{rgb}{0.0,0.0,0.0};
  \pgfsetstrokecolor{strokecol}
  \draw (114.87bp,227.81bp) ellipse (18.0bp and 18.0bp);
\end{scope}
  % Node: 8
\begin{scope}
  \definecolor{strokecol}{rgb}{0.0,0.0,0.0};
  \pgfsetstrokecolor{strokecol}
  \draw (148.38bp,97.12bp) ellipse (18.0bp and 18.0bp);
\end{scope}
%
\end{tikzpicture}

\caption{A universal graph for the family of all
trees with 6 vertices}
\label{fig:trees}
\end{figure}


\chapter{Introduction}
\label{c:intro}

\newcommand{\ChapterRef}[1] {\Cref{#1}: \nameref{#1}}

A \emph{graph} is a collection of items, some pairs of which are related.  We
call the items \emph{vertices} and the pairwise relationships \emph{edges}.
Abstractly, we can view a graph as a collection of dots (the vertices) with
some pairs of dots connected by a line (the edges).  This simple abstraction
is used to model systems from a huge variety of domains: a molecule is a
collection of atoms joined together by bonds \citep{sussenguth1965graph}; an
image may be summarised using a vertex for each coloured region with the
vertices for adjacent regions represented by edges
\citep{DBLP:conf/icip/OlatunbosunDE96}.  We may wish to determine whether a
copy of all
or a large part of a given object---such as a molecule or image---is contained
in another given object.  The goal of this dissertation is to introduce new,
practical algorithms for such problems.

We first consider the \emph{maximum common induced subgraph} family of
problems, in which we seek a large subgraph that is contained in each of two
given graphs.  Depending on the flavour of the problem, these graphs may be
directed or undirected and may or may not have labels.  All of these variants
of the can be handled by the \McSplit\ family of algorithms.  These algorithms
use a simple, fast and space-efficient data structure to keep track of the set
of vertices in the second graph to which a given vertex in the first graph may
be mapped.

The second problem we consider is \emph{induced subgraph isomorphism}: to
determine whether a given graph (the ``target graph'') contains another given
graph (the ``pattern graph'') as an induced subgraph.  We again use a variant
of the \McSplit\ algorithm---\McSplit-SI---to solve the problem. This version
of the algorithm has a specialised version of \McSplit's data structure that
allows very fast processing of sparse graphs.

For the final problem considered in this dissertation, we turn from subgraphs
to supergraphs.  We study the problem of finding, for a given family of graphs,
an \textit{induced universal graph}---that is, a graph that contains every
member of the family as an induced subgraph---with as few vertices as
possible.  This problem generalises the \textit{minimum common supergraph}
problem to an arbitrary number of input graphs.  Although much progress has
been made on asymptotic results on the size of induced universal graphs, almost no
work has been done on developing algorithms to solve the problem exactly.  This
dissertation presents an algorithm for finding minimal induced universal graphs
using \McSplit-SI as a subroutine, and presents new terms of integer sequences
generated using the program.  Further, we present a hill-climbing method for
finding small (although possibly not optimal) induced universal graphs.

\section{Structure of the Dissertation}

The remainder of this dissertation is structured as follows.

\paragraph*{\ChapterRef{c:background}} provides context for this dissertation's
contributions: definitions, a survey of related work, and a description of the
experimental setup.

\paragraph*{\ChapterRef{c:mcsplit-i-undirected}} introduces the \McSplit\
algorithm for the maximum common induced subgraph problem and its variants, and
carries out experimental comparisons with existing state of the art solvers.
The chapter has a detailed discussion of the similarities and differences
between \McSplit\ on one hand and constraint programming and clique approaches
on the other.


\paragraph*{\ChapterRef{c:swapping-graphs-mcsplit}} asks whether and when
it is useful to swap the two graphs given as input to a \McSplit.  We develop
two simple rules for determining when to swap the graphs, and generalise
these to the algorithm \McSplit-2S which branches on vertices of both
input graphs.

\paragraph*{\ChapterRef{c:mcsplit-si}} introduces the \McSplit-SI algorithm for the
induced subgraph isomorphism problem.  This is based on \McSplit, but uses
special data structures designed to work well if the input graphs are sparse.
We present an extensive set of experiments comparing the algorithm to state of
the art solvers.  Finally, we modify \McSplit-SI to solve the maximum common
induced subgraph problem, and show that it outperforms the state of the art
solver for large, sparse graphs.

\paragraph*{\ChapterRef{c:mcsplit-i-undirected}} introduces exact and heuristic
algorithms for the minimum induced universal graph problems, and uses these
algorithms to add new terms to integer sequences.  The chapter includes
an experimental study of ordering heuristics for these algorithms.

\paragraph*{\ChapterRef{c:conclusion}} concludes.


%%  %==============================================================================
%%  \section{Thesis Statement}
%%  \label{c:intro:thesisstatement}
%%  
%%  TODO


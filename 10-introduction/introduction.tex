\chapter{Introduction}
\label{c:intro}

\newcommand{\ChapterRef}[1] {\Cref{#1}: \nameref{#1}}

\section{Background and Motivation}
A \emph{graph} is a collection of items, some pairs of which are related; we
call the items \emph{vertices} and the pairwise relationships \emph{edges}.
%Abstractly, we can view a graph as a collection of dots (the vertices) with
%some pairs of dots connected by a line (the edges).
This simple abstraction
is used to model systems from a huge variety of domains: a molecule is a
collection of atoms joined together by bonds \citep{sussenguth1965graph}; an
image may be summarised using a vertex for each coloured region with the
vertices for adjacent regions joined by edges
\citep{DBLP:conf/icip/OlatunbosunDE96}.  We may wish to determine whether a
copy of all
or a large part of a given object---such as a molecule or image---is contained
in another given object.  The goal of this dissertation is to introduce new,
practical algorithms for such problems.

We first consider the \emph{maximum common induced subgraph} family of
problems, in which we seek a large graph that appears as part (that is,
is a \emph{subgraph}) of each of two
given graphs.  The problem has a number of variants, such as a version where
the vertices have labels---atomic number in the case of molecules---and
the subgraphs must have the same labels.
The \McSplit\ family of algorithms can handle many of these variants.  These algorithms
use a simple, fast and space-efficient data structure to keep track of the set
of vertices in the second graph to which a given vertex in the first graph may
be mapped.

The second problem we consider is \emph{induced subgraph isomorphism}: to
determine whether a given graph (the ``target graph'') contains another given
graph (the ``pattern graph'') as an induced subgraph.  We again use a variant
of the \McSplit\ algorithm---\McSplit-SI---to solve the problem. This version
of the algorithm has a specialised version of \McSplit's data structure that
allows very fast processing of sparse graphs.

Finally, we study the problem of finding, for a given family of graphs,
an \textit{induced universal graph}---that is, a graph that contains every
member of the family as an induced subgraph---with as few vertices as
possible.
Although much progress has
been made on asymptotic results on the size of induced universal graphs, almost no
work has been done on developing algorithms to solve the problem exactly.  This
dissertation presents an algorithm for finding minimal induced universal graphs
using \McSplit-SI as a subroutine, and presents new terms of integer sequences
generated using the program.  Further, we present a hill-climbing method for
finding small (although possibly not optimal) induced universal graphs.

\section{Structure of the Dissertation}

The remainder of this dissertation is structured as follows.

\paragraph*{\ChapterRef{c:background}} provides context for this dissertation's
contributions: definitions, a survey of related work, and a description of the
experimental setup.

\paragraph*{\ChapterRef{c:mcsplit-i-undirected}} introduces the \McSplit\
algorithm for the maximum common induced subgraph problem and its variants, and
carries out experimental comparisons with existing state of the art solvers.
We find that the algorithm is more than an order of magnitude faster than the prior
state of the art for unlabelled graphs, including in the variant of the problem
where the subgraph must be connected.
The chapter has an empirical study of ordering strategies for the vertices
of the two input graphs, and a detailed study---both in theory and practice---of
the similarities and differences
between \McSplit\ on one hand, and constraint programming and clique approaches
on the other.


\paragraph*{\ChapterRef{c:swapping-graphs-mcsplit}} asks whether and when it is
useful to swap the two graphs given as input to \McSplit.  We develop two
simple and effective rules for determining when to swap the graphs, and
generalise these to the algorithm \McSplit-2S which branches on vertices of
both input graphs.

\paragraph*{\ChapterRef{c:mcsplit-si}} introduces the \McSplit-SI algorithm for the
induced subgraph isomorphism problem.  This uses
special data structures designed to work well if the input graphs are sparse.
We present an extensive set of experiments comparing the algorithm to state of
the art solvers; these experiments demonstrate that the algorithm runs faster than the
best existing solvers on several classes of random and structured
graphs.
In addition, we demonstrate that \McSplit-SI achieves generalised arc consistency
on the all-different constraint, and we introduce two variants of the algorithm
including one optimised for dense graphs.
Finally, we modify \McSplit-SI to solve the maximum common
induced subgraph problem, and show that this new version outperforms the state of the art
solver for large, sparse graphs.

\paragraph*{\ChapterRef{c:universal-graphs}} introduces exact and heuristic
algorithms for the minimal induced universal graph problem, and uses these
algorithms to add new terms to entries in
the On-Line Encyclopedia of Integer Sequences.  The chapter includes
an experimental study of ordering heuristics for these algorithms.

\paragraph*{\ChapterRef{c:conclusion}} summarises progress made in the dissertation
and outlines directions for future work.

\section{Publications and Authorship}

During the course of my PhD I carried out work on other graph algorithms---in particular,
maximum weight clique and treedepth---that is not presented in this thesis.
For completeness, the following is a full list of publications
since the start of my PhD.

\begin{enumerate}
    \item\bibentry{DBLP:conf/ijcai/McCreeshPT17}
    \item\bibentry{DBLP:conf/cp/McCreeshPST17}
    \item\bibentry{DBLP:conf/cpaior/HoffmannMNPRS018}
    \item\bibentry{DBLP:journals/jair/McCreeshPST18}
    \item\bibentry{DBLP:conf/cpaior/ArchibaldDHMP019}
    \item\bibentry{DBLP:conf/wea/000120a}
    \item\bibentry{DBLP:conf/iwpec/000120}
    \item\bibentry{DBLP:conf/iwpec/000120a}
    \item\bibentry{DBLP:conf/gg/McCreeshP020}
    \item\bibentry{DBLP:conf/cp/GochtMMNPT20}
    \item\bibentry{DBLP:journals/cor/DelormeGGKMPT22}
\end{enumerate}

\Cref{c:mcsplit-i-undirected} of this dissertation contains material from the
first publication on the list.  I designed and implemented the \McSplit\
algorithm and was the lead author of this paper, which was
co-authored by Ciaran McCreesh and Patrick Prosser.
Ciaran McCreesh carried out the experimental comparison of \McSplit\ with
existing solvers.  \Cref{sec:mcsplit-experiments}, which uses the data from these
experiments, is a revised and greatly extended version of the experimental
section of the paper; the original section was jointly written by all three
of the paper's authors.

I am the sole author of all other chapters of this dissertation.

\section{Earlier Publications by the Author}

The following is a list of the author's publications prior to starting
work on this dissertation.

\begin{enumerate}
    \item\bibentry{DBLP:conf/sigecom/DickersonMPST16}
    \item\bibentry{DBLP:conf/ijcai/McCreeshPT16}
    \item\bibentry{DBLP:conf/cp/McCreeshPT16}
    \item\bibentry{DBLP:journals/constraints/ManloveMT17}
\end{enumerate}

%%  %==============================================================================
%%  \section{Thesis Statement}
%%  \label{c:intro:thesisstatement}
%%  
%%  TODO


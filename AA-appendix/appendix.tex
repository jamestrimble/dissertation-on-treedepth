\chapter{MCSa2 upper bound for maximum common induced subgraph}
\label{a:appendix-mcsa2}

TODO: write this. The figures compare MCSa2 bound with McSplit bound and with MCSa1 bound.
The MCSa2 bound is very similar to the MCSa1 bound on average, and the comparison with the McSplit bound
looks like the MCSa1 comparison.  But on an instance-by-instance basis, MCSa1 often has a much stronger
bound than MCSa2 or vice versa.  It would be interesting to see if we could do a portfolio of clique
algorithms (future work).

\begin{figure}[h!]
    \centering
    \includegraphics*[width=0.65\textwidth]{14-mcsplit-i-undirected/bound-experiments/plots/bounds-mcsa2-plot.pdf}
    \caption{TODO: make this different from the corresponding caption for MCSa1 in the McSplit chapter.
        Probably refer back to that caption.
        As the number of vertex labels is increased, the clique algorithm's bound
    	becomes stronger than the McSplit bound.
        The vertical axis shows the ratio of initial McSplit upper bound
	to initial clique algorithm (MCSa2) upper bound for 1800 instances, with one dot per instance;
	for values less than 1, McSplit's bound is better than that of the clique algorithm.
	The horizontal axis shows the maximum label $m$; each vertex label is a random integer 
	from the range $[1,m]$.}
    \label{figure:bound-mcsa2}
\end{figure}

\begin{figure}[h!]
    \centering
    \includegraphics*[width=0.65\textwidth]{14-mcsplit-i-undirected/bound-experiments/plots/bounds-mcsa1vs2-plot.pdf}
    \caption{TODO write this}
    \label{figure:bound-mcsa1vs2}
\end{figure}


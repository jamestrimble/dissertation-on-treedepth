\chapter{A two-sided look at \McSplit}
\label{c:swapping-graphs-mcsplit}

\section{Introduction}

Given graphs $G$ and $H$, it is clear that a maximum common subgraph of $H$ and $G$ is
also a maximum common subgraph of $G$ and $H$.  In this chapter, we ask whether
it is ever useful to swap the two input graphs when calling a maximum common subgraph
solver.  We will see that swapping the graphs can have a very large effect on the
execution time of \McSplit\
if the two graphs differ in density or order.  Finally, we introduce a modified
version of \McSplit\ that generalises the idea of swapping graphs.

\section{When should we swap the graphs?}

We begin by plotting run times of \McSplit\ with and without swapping graphs, to
give an idea of the difference that swapping can make.  We refer to the swapping
version of \McSplit\ as \McSplit-Swap; this is shown in
\Cref{McSplitSwapAlg}.

\begin{algorithm}[h!]
\DontPrintSemicolon
\nl $\FuncSty{McSplit-Swap}(G, H)$ \label{McSplitSwapFun} \;
\nl \Begin{
    \nl $M \gets \FuncSty{McSplit}(H,G)$ \LeftComment{Call \McSplit\ with the graphs swapped}\;
    \nl $\KwSty{return}$~$\{(w, v) \mid (v, w) \in M\}$ \LeftComment{Reverse the mapping}
}
    \caption{\McSplit-Swap: a version of \McSplit\ that swaps the input graphs.} 
\label{McSplitSwapAlg}
\end{algorithm}

\Cref{subfig:runtime-swapping-scatter-mcsplain}
shows run times for the MCS Plain instances.  The horizontal axis shows run time
without swapping the graphs; the vertical axis shows run time with swapping.  Although swapping
graphs can make a small difference, it does not change run time by as much as an order of 
magnitude for any of these instances.

\begin{figure}[h!]
    \centering
    \subfigure[][MCS Plain instances] {
        \centering
        \includegraphics*[width=0.44\textwidth]{14-mcsplit-i-undirected/modified-mcsplit-experiment/plots/plots/left-vs-right-mcsplain}
        \label{subfig:runtime-swapping-scatter-mcsplain}
    }
    \subfigure[][Random 24-vertex instances] {
        \centering
        \includegraphics*[width=0.44\textwidth]{14-mcsplit-i-undirected/modified-mcsplit-experiment/plots/plots/left-vs-right-random2}
        \label{subfig:runtime-swapping-scatter-random2}
    }
    \caption{Run times in ms for \McSplit\ (horizontal axis) and \McSplit\ with graphs $G$ and $H$ swapped
        (vertical axis).  Each point represents one graph pair.  Swapping the graphs has little effect on run time
        for MCS Plain instances, but changes the run time by orders of magnitude
        for many of the random instances.}\label{figure:runtime-swapping-scatter}
\end{figure}

Within each MCS Plain instance, the two graphs are very similar: they are produced
using the same graph generator, have the same number of vertices, and have very similar density.
(See \Cref{figure:mcsplain-densities} in the appendix for a scatter plot of densities.)
Thus, it is unsurprising that swapping the graphs has little effect on run time.
To examine pairs of graphs with very different characteristics, we generated two
sets of random instances using the Erdos-Renyi $G(n,p)$ model.  The first set has
24 vertices per graph, with the $p$ parameter of the generator chosen randomly from
$\{0.01, 0.02, \dots, 0.99\}$; thus, the graphs have the same order but vary
greatly in density.  The second set of instances has the density parameter
$p$ set to $0.3$ for all instances, with the number of vertices in each graph randomly
chosen from $\{10, 11, \dots, 40\}$.

\Cref{subfig:runtime-swapping-scatter-random2} shows run times with and without swapping
for the first set of random graphs.  For these instances, unlike the MCS Plain instances,
there are very clearly some instances for which swapping graphs is beneficial, and others
for which swapping greatly increases the run time.

\subsection{Random graphs with fixed $n$ and varying density}

Can we tell in advance whether we should swap the graphs of a given instance?
\Cref{figure:coloured-scatter-run-times-density} shows that, in the case of our
random 24-vertex instances, there is a very strong association between the densities
of the two graphs and whether we should swap graphs.  In this plot, the axes
measure graph density,\footnote{By \emph{density}, we are referring to the actual density of a graph,
$\frac{|E(G)|}{n_G(n_G-1)/2}$, rather than the parameter $p$ used to generate
the graph. In practice the two measures are very similar.}. Colour is used to show
the effect of swapping graphs on run time; red dots represent those instances
that are solved more quickly with the graphs swapped.  The diagonal lines on the plot
are $y=x$ and $y=1-x$; these divide the plot into four triangular regions.
The figure shows clearly that instances lying
in the upper and lower triangles tend to be solved more quickly by \McSplit, while
those lying in the left and right triangles tend to be solved more
quickly by \McSplit-Swap.

\begin{figure}[h!]
    \centering
    \subfigure[][Density] {
        \centering
        \includegraphics*[width=0.44\textwidth]{14-mcsplit-i-undirected/modified-mcsplit-experiment/plots/plots/density-when-swap}
        \label{figure:coloured-scatter-run-times-density}
    }
    \subfigure[][``Extremeness'' of density] {
        \centering
        \includegraphics*[width=0.44\textwidth]{14-mcsplit-i-undirected/modified-mcsplit-experiment/plots/plots/density-extremeness-when-swap}
        \label{figure:coloured-scatter-run-times-extremeness}
    }
    \caption{There is a strong association between the densities of graphs $G$ and $H$
        and whether it is beneficial to use \McSplit-Swap rather than \McSplit.
        The first subfigure plots density of $G$ against density of $H$, with one point
        plotted for each of the random 24-vertex instances.  Instances for which \McSplit-Swap
        is faster are plotted in red; instances where \McSplit\ is faster
        are plotted in blue.  Dark red and dark blue indicate that \McSplit-Swap and \McSplit,
        respectively, result in run times at least twice as fast as the alternative.}
        \label{figure:coloured-scatter-mcis-run-times}
\end{figure}

The union of the left and right triangles in
\Cref{figure:coloured-scatter-run-times-density} has a simple characterisation.
Define the measure \emph{density extremeness} of
a graph as $\left|\frac{1}{2} - d\right|$, where $d$ is the graph's
density.  This is simply a measure of how close a graph's
density is to either 0 or 1; thus, a clique and an independent set
have the highest possible density extremeness.  The union of the left and right triangles
in \Cref{figure:coloured-scatter-run-times-density} contains exactly those
graph pairs $(G,H)$ such that the density extremeness of $G$ is greater than
the density extremeness of $H$.
The second subfigure, \Cref{figure:coloured-scatter-run-times-extremeness},
replots the data in \Cref{figure:coloured-scatter-run-times-density},
with density extremeness rather than density measured on the axes.
It is evident from the figure that for instances such that the density extremeness of $G$
exceeds the density extremeness of $H$, \McSplit-Swap almost always
runs faster than \McSplit.

Given the strong relationship between the densities of $G$ and $H$ and the 
optimal order in which to pass the graphs to \McSplit, a natural next
step is to devise a version of \McSplit\ such that the graphs are
swapped if and only if the density extremeness of $G$ is greater
than the density extremeness of $H$.  We call this algorithm
\McSplit-SD; it is shown in \Cref{McSplitSDAlg}.

\begin{algorithm}[h!]
\DontPrintSemicolon
\nl $\FuncSty{McSplit-SD}(G, H)$ \label{McSplitSDFun} \;
\nl \Begin{
    \nl \If{$\left|\frac{1}{2} - d_G\right| > \left|\frac{1}{2} - d_H\right|$}{
        \nl $\KwSty{return}$~$\FuncSty{McSplit-Swap}(G,H)$ \;
    }
    \nl $\KwSty{return}$~$\FuncSty{McSplit}(G,H)$
}
    \caption{\McSplit-SD: a version of \McSplit\ that uses density to decide whether to swap the input graphs.} 
\label{McSplitSDAlg}
\end{algorithm}

\Cref{figure:left-vs-smart-d-mcis} compares on our set of 24-vertex random instances
the run times of three solvers: \McSplit, \McSplit-SD, and the virtual best solver (VBS) of
\McSplit\ and \McSplit-Swap.  The cumulative plot shows that \McSplit-Swap and the VBS
have almost identical performance overall, and that both outperform \McSplit.  The scatter
plot in \Cref{figure:left-vs-smart-random2} shows that \McSplit-SD is never much slower than
\McSplit\ on these instances, and is orders of magnitude faster on several of the
instances.  In summary, for this family of instances, \McSplit-SD clearly improves upon
\McSplit, and makes near-perfect decisions when selecting between \McSplit\ and \McSplit-Swap.

\begin{figure}[h!]
    \centering
    \subfigure[][Cumulative plot of instances solved] {
        \centering
        \includegraphics*[width=0.52\textwidth]{14-mcsplit-i-undirected/modified-mcsplit-experiment/plots/plots/mcsplit-random-smart-density-cumulative}
        \label{figure:left-vs-smart-density-cumulative}
    }
    \subfigure[][Run times (ms)] {
        \centering
        \includegraphics*[width=0.41\textwidth]{14-mcsplit-i-undirected/modified-mcsplit-experiment/plots/plots/left-vs-smart-random2}
        \label{figure:left-vs-smart-random2}
    }
    \caption{The run times of \McSplit-SD are faster overall than those of \McSplit\ for the
        random 24-vertex instances, and almost indistinguishable from those of the virtual
        best solver of \McSplit\ and \McSplit\ with swapped graphs.}
        \label{figure:left-vs-smart-d-mcis}
\end{figure}

\subsection{Random graphs with similar density and varying $n$}

We now turn to our family of random instances generated with $p=0.3$ and varying values of
the order parameter $n$.  \Cref{figure:order-when-swap} shows a strong relationship between
the graphs' densities and the relative run times of \McSplit\ and \McSplit-Swap: \McSplit\
is generally preferable if $G$ has fewer vertices than $H$, and \McSplit-Swap is preferable
if $G$ has more vertices than $H$.

\begin{figure}[h!]
    \centering
    \includegraphics*[width=0.55\textwidth]{14-mcsplit-i-undirected/modified-mcsplit-experiment/plots/plots/order-when-swap}
    \caption{For our random $p=0.3$ instances, is is preferable to use \McSplit\ when $G$
        has fewer vertices than $H$, and to use \McSplit-Swap when $G$ has more vertices than $H$.
        The plot shows one point per instance.  Instances for which \McSplit-Swap
        is faster are plotted in red; instances where \McSplit\ is faster
        are plotted in blue.  Dark red and dark blue indicate that \McSplit-Swap and \McSplit,
        respectively, result in run times at least twice as fast as the alternative.}
    \label{figure:order-when-swap}
\end{figure}

\McSplit-SO, an algorithm that swaps the graphs if the order of $G$ is greater than the
order of $H$, is shown in \Cref{McSplitSOAlg}.
\Cref{figure:left-vs-smart-o-mcis} shows the run times of \McSplit, \McSplit-SO,
and the VBS of \McSplit\ and \McSplit-Swap for our $p=0.3$ instances.  As in our earlier
experiment with \McSplit-SD, the \McSplit-SO algorithm performs about as well as the
VBS, and is not substantially slower than \McSplit\ on any instance.  However, the improvement
provided by \McSplit-SO is less dramatic than that provided by \McSplit-SD; \McSplit-SO
is seldom more than an order of magnitude faster than \McSplit\ on non-trivial instances.

\begin{algorithm}[h!]
\DontPrintSemicolon
\nl $\FuncSty{McSplit-SO}(G, H)$ \label{McSplitSOFun} \;
\nl \Begin{
    \nl \If{$n_G > n_H$}{
        \nl $\KwSty{return}$~$\FuncSty{McSplit-Swap}(G,H)$ \;
    }
    \nl $\KwSty{return}$~$\FuncSty{McSplit}(G,H)$
}
    \caption{\McSplit-SO: a version of \McSplit\ that uses vertex counts to decide whether to swap the input graphs.} 
\label{McSplitSOAlg}
\end{algorithm}

\begin{figure}[h!]
    \centering
    \subfigure[][Cumulative plot of instances solved] {
        \centering
        \includegraphics*[width=0.52\textwidth]{14-mcsplit-i-undirected/modified-mcsplit-experiment/plots/plots/mcsplit-random-smart-order-cumulative}
        \label{figure:left-vs-smart-order-cumulative}
    }
    \subfigure[][Run times (ms)] {
        \centering
        \includegraphics*[width=0.41\textwidth]{14-mcsplit-i-undirected/modified-mcsplit-experiment/plots/plots/left-vs-smart-random3}
        \label{figure:left-vs-smart-random3}
    }
    \caption{The run times of \McSplit-SD are faster overall than those of \McSplit\ for the
        random 24-vertex instances, and almost indistinguishable from those of the virtual
        best solver of \McSplit\ and \McSplit\ with swapped graphs.}
        \label{figure:left-vs-smart-o-mcis}
\end{figure}

\section{Generalising \McSplit-SD and \McSplit-SO}

Why are \McSplit-SD and \McSplit-SO effective?  TODO: write about opposite of smallest domain first.  Do cumumlative curves
for two random families and mcsplain with mcsplit-f, mcsplit-F, mcsplit.  Show that mcsplit-F helps a bit for random families
and doesn't require knowledge of either order nor density.  Give negative result for mcsplain: mcsplit-F doesn't seem to help
much, but scatter shows that mcsplit-f is a little bit bad.

\section{Things to write about}

\begin{itemize}
    \item McSplit-F
    \item K and H algorithm - show that swapping is useful
    \item Same for Versari algorithm?
\end{itemize}

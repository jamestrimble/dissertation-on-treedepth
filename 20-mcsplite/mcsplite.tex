\chapter{The Maximum Common Edge Subgraph Problem}
\label{c:mcsplite}

\section{McSplit-L}

?? Write about how we can use McSplit on the line graphs to solve MCES.

\section{A CP model} \label{sec:mces-cp-model}

Before introducing our CP model for MCES, we define some notation.  Let $P$ be the pattern graph,
and $T$ the target graph.  We assume that the vertex set of each graph contains only integers.
Let $A_P = \{(u,v) \mid \{u,v\} \in E(P) \wedge u < v \}$ and
    $A_T = \{(u,v) \mid \{u,v\} \in E(P) \}$.
Thus, $A_P$ contains an oriented edge for each edge in the pattern graph, and $A_T$ contains
oriented edges in both directions for each edge in the target graph.

We assume that the pattern graph has no more vertices than the target graph.  If this is not
the case, the two graphs should be swapped before encoding the instance using the CP model.

In our CP model, we have a variable $e_{(v,w)}$ for each oriented edge $(v,w) \in A_P$,
and each of these variables has domain $A_T \cup \bot$.  If $e_{(v,w)}$ takes the value $(t,u)$,
then edge $\{v,w\}$ in the pattern graph is mapped to edge $\{t,u\}$ in the target graph, and vertices
$v$ and $w$ in the pattern graph are mapped to $t$ and $u$ respectively in the target graph.
If $e_{(v,w)}$ takes the value $\bot$, the edge $\{v,w\}$ is not used in the mapping (and is thus not
part of the common subgraph).

For each vertex $v$ in $V(P)$, we have a variable $m_v$, and each of these variables has domain $V(T)$.
if $m_v$ takes the value $w$, this represents that vertex $v$ in the pattern graph is mapped to vertex
$w$ in the target graph.

Our model is an CP optimisation problem, the objective of which is to maximise
the number of $e_{(v,w)}$ variables that take non-$\bot$ values (and thus, to maximise the number of
edges in the common subgraph).

To ensure that each vertex in the target graph has no more than one vertex in the pattern graph mapped
to it, we post
\[
\textit{alldifferent}(\{m_v \mid v \in V(P)\}).
\]

To ensure that the vertex mappings are consistent with the edge mappings, we post
\begin{equation}
\label{implicationconstraint}
\forall (v,w) \in A_P, (t,u) \in A_T : \quad e_{(v,w)} = (t,u) \implies m_v=t \wedge m_w=u.
\end{equation}

In each of these constraints, the implication ($\implies$) may be replaced with a two-directional
implication ($\Longleftrightarrow$).  This would (I think) provide stronger filtering, but is not
compatible with the data structures that we will describe for our McSplit-E
algorithm.\footnote{The leftward implication ($\Longleftarrow$)
is implied by the rest of the model, and therefore its omission does not make the model incorrect.
To see this, suppose that in a complete instantiation that maximises the objective value
we have values $t,u,v,w$ such
that $m_v=t \wedge m_w=u$.  Then there are three mutually exclusive and exhaustive possibilities for the
value of $e_{(v,w)}$: it may be $\bot$, $(t,u)$, or some element of $A_T$ other than $(t,u)$.
The first of these possibilities is impossible since the objective
value may be increased by one without violating any constraints by giving the value $(t,u)$ to $e_{(v,w)}$;
thus, we violate our assumption that the objective is maximised.
The third possibility is also impossible since, by equation \ref{implicationconstraint}, we have
either $m_v=t$ or $m_w=u$.  Therefore, it must be the case that $e_{(v,w)} = (t,u)$
and thus the leftward implication holds.}

Ndiaye and Solnon \cite{DBLP:conf/cp/NdiayeS11} propose a similar model for MCES on directed graphs,
which can be used to solve the problem on an undirected graph by replacing each edge with directed
edges in both directions.  Their model does not include variables for vertex mappings, but uses binary
constraints between pairs of edge variables to ensure that incidence relationships are preserved
by the edge mapping.  ?? Mention their use of soft all-diff.

%We have implemented three constraint program models for MCES in the MiniZinc language.
%In each case, we assume without loss of generality that the pattern graph has no more vertices than the target graph.
%
%?? Write about models 1 and 2.

\section{A MiniZinc model}

\Cref{fig:model3} presents MiniZinc code for the model described in the
previous section.  The variables \texttt{m} in the code correspond to the
$m_v$ variables in CP model, and the variables \texttt{m\_edge} correspond
to the $e_{(v,w)}$ variables.

On the line beginning \texttt{constraint forall}, the \texttt{->} implication
may be replaced by \texttt{<->} without affecting correctness, as described
in the previous section.

?? Say something about the bound being weak.  ?? How can we improve
this?---a soft all-diff propagator?

\begin{figure}[htb]
\vspace{.8em}
\scriptsize
\begin{verbatim}
int: np;                 % order of pattern graph
int: nt;                 % order of target graph

int: mp;                 % size of pattern graph
int: mt;                 % size of target graph * 2
                         %   (i.e. number of oriented edges)

set of int: VP = 1..np;  % pattern vertices
set of int: VT = 1..nt;  % target vertices

array[VP, VP] of int: P; % adjacency matrix of pattern graph
array[VT, VT] of int: T; % adjacency matrix of target graph

array[1..mp, 1..2] of int: PE; % edge list of pattern graph
array[1..mt, 1..2] of int: TE; % oriented edge list of target graph

array[VP] of var VT: m;  % pattern vertex -> target vertex mappings
array[1..mp] of var 0..mt: m_edge;  % pattern edge index -> target
                                    %   edge index mappings. 0 means _|_

var 1..mp: objval;

constraint forall (i in 1..mp, j in 1..mt)
        (m_edge[i]==j -> (m[PE[i,1]]==TE[j,1] /\ m[PE[i,2]]==TE[j,2]));
constraint objval = sum (a in m_edge) (a != 0);
constraint alldifferent(m);

solve :: int_search(m_edge, first_fail, indomain_split, complete)
        maximize objval;
\end{verbatim}
\vspace{-1em}
\caption{A MiniZinc model for MCES}\label{fig:model3}
\end{figure}

\section{McSplit-E}

We now introduce our second dedicated solver for MCES (after McSplit-L): McSplit-E.  Whereas
McSplit-L solves MCES indirectly by solving MCIS on the line graphs, McSplit-E
models MCES directly, exploring essentially the same search tree as the CP
model in \cref{sec:mces-cp-model} with forward checking.  Unlike a typical CP
solver, McSplit-E uses the compact data structures of McSplit
(albeit on edge rather than vertex variables) and uses a similar partitioning
algorithm to filter domains.

As in the CP model, we have a variable for each pattern edge and a value
for each \textit{oriented} target edge.  Edges are stored as ordered pairs; thus the mapping
$(1,2) \rightarrow (3,4)$ indicates not only that edge $\{1,2\}$ of the pattern graph is
mapped to edge $\{3,4\}$ of the target graph, but also that vertices $1$ and $2$ are mapped
to vertices $3$ and $4$ respectively (rather than the other way round).

In McSplit-E, all branching decisions are made on edge variables rather than vertex variables.
At any branching level of the search tree, a pattern-edge variable is chosen, and all of the possible
oriented target-edge values are tried in turn, followed by the value
$\bot$.\footnote{We could, instead, try branching on vertex assignments, and this is
something that would be worth exploring empirically in future.  However, some other
branching schemes such as branching on the binary decision of whether an edge variable takes
a particular value would not allow us to use the compact data structure for domains.}

We use the example graphs in \cref{fig:splitp-example} to illustrate McSplit-E's data structures.
\begin{figure}[htb]
    \centering
        \begin{tikzpicture}[scale=.4]%{{{
        \node[draw, circle, fill=white, inner sep=1.4pt, font=\bfseries] (Na) at (0,  0)
        {1};
        \node[draw, circle, fill=white, inner sep=1.4pt, font=\bfseries] (Nb) at (1, -2)
        {2};
        \node[draw, circle, fill=white, inner sep=1.4pt, font=\bfseries] (Nc) at (2, 0)
        {3};

        \draw [edge] (Na) -- (Nb);
        \draw [edge] (Nb) -- (Nc);
        \draw [edge] (Na) -- (Nc);

    \end{tikzpicture}
    \quad
    \begin{tikzpicture}[scale=.4]%{{{
        \node[draw, circle, fill=white, inner sep=1.4pt, font=\bfseries] (Na) at (0,  0)
        {$a$};
        \node[draw, circle, fill=white, inner sep=1.4pt, font=\bfseries] (Nb) at (2, 0)
        {$b$};
        \node[draw, circle, fill=white, inner sep=1.4pt, font=\bfseries] (Nc) at (1, -1.5)
        {$c$};
        \node[draw, circle, fill=white, inner sep=1.4pt, font=\bfseries] (Nd) at (1, -3)
        {$d$};

        \draw [edge] (Na) -- (Nb);
        \draw [edge] (Na) -- (Nc);
        \draw [edge] (Nb) -- (Nc);
        \draw [edge] (Nc) -- (Nd);

    \end{tikzpicture}


    \caption{Example graphs $P$ and $T$}
    \label{fig:splitp-example}
\end{figure}

Initially, each pattern edge has all four
target edges and $\bot$ in its domain.  Moreover, each pattern edge can be mapped to a target edge in either orientation
(thus, in our example, $(1,2) \rightarrow (a,b)$ and $(1,2) \rightarrow (b,a)$ are both valid mappings). 
We could store the three domains separately (for example, with an identical nine-element list for each domain).
Rather than doing this, we store the same
information using two arrays and an integer.  The first array is a list of pattern edges;
the second array is a list of target edges. Finally, we store the integer 2 to signify that either orientation of a
target edge may be chosen (for example, edge $(1,2)$ may be mapped to either $(a,b)$ or $(b,a)$).
Thus, before any edge assignments are made, the domains are stored as:

\begin{align*}
    [(1,2), (1,3), (2,3)] & \qquad [(a,b), (a,c), (b,c), (c,d)] & 2
\end{align*}

Now, suppose that the algorithm has made the tentative assignment of pattern edge $(1,2)$ to target edge $(a,c)$.
The forward-checking algorithm reduces the domain of $m_1$ to the singleton $\{a\}$, and reduces
the domain of $m_2$ to the singleton $\{c\}$.  We then instantiate $m_1$ and $m_2$, in turn, to
the unique values in their domain, and filter the edge-variable domains accordingly.  As a result,
edge $(1,3)$ is left with $(a,b)$ and $\bot$ in its domain, while edge $(2,3)$ has $(c,b)$, $(c,d)$
and $\bot$ in its domain.  These domains are stored as follows (with the 1 at the end of
each line signifying that only the shown orientation of each target edge is permitted).

\begin{align*}
    [(1,3)] & \qquad [(a,b)]        & 1 \\
    [(2,3)] & \qquad [(c,b), (c,d)] & 1
\end{align*}

As the algorithm runs, the domain of each $m_v$ variable contains either the full set $V(T)$ or
a single vertex determined by an edge-mapping decision.  Therefore, it is not necessary to explicitly
store the domains of vertex-mapping variables.

We can view the McSplit-L approach (where values correspond to vertices in the target line graph, and
thus effectively correspond to unoriented edges in the original target graph)
as delaying the decision of which way to orient the target
edges until after finding an optimal solution.
McSplit-E, by contrast, makes orientation decisions as early as possible.

We now describe McSplit-E (pseudocode for which appears in \cref{McSplitAlg})
in more detail.
We begin our description with the $\FuncSty{Refine}$ function.  Given a set of label-classes
$\AlgVar{future}$ and a mapping from pattern vertex $v$ to target vertex $t$, this function
creates a refined set of label-classes that results from the vertex mapping,
any also moves implied edge-edge mappings to $E$.  Each label-class is split into at most
two new label classes---one class for edges that include $v$ (in the pattern graph) or $t$
(in the target graph), and one for edges that do not contain $v$ or $t$.  If both sets in a new
label-class are non-empty, it is included in $\AlgVar{future'}$.  \Lineref{AddToE} deals with the
special case where the edges contain $v$ or $t$, and the other endpoints have already been mapped.
In this case, it is safe to move the single edge mapping in this label-class to $E$.\footnote{Alternatively,
we could put edge mappings in $E$ in the $\FuncSty{Search}$ function instead, when we make a mapping decision
in the for-loop.  In practice, this is less efficient.  ?? why?}

\begin{algorithm}[htb]
\scriptsize
\DontPrintSemicolon
\nl $\FuncSty{EdgesIncidentToVertex}(E,v)$ \;
\nl \Begin{
\nl   $E' = \emptyset$ \;
\nl   \For{$(u,w) \in E$}{
\nl     \LeftComment{If this edge is incident to $v$, add the edge to $E'$ with $v$ stored as the first vertex.} \;
\nl     \lIf{u=v}{$E' = E' \cup (u,w)$}
\nl     \lElseIf{w=v}{$E' = E' \cup (w,u)$}
}
\nl   $\KwSty{return}$~$E'$ \;
}
\;
\nl $\FuncSty{EdgesNotIncidentToVertex}(E,v)$ \;
\nl \Begin{
\nl   $\KwSty{return}$~$\{(u,w) \in E \mid u\not=v \wedge w\not=v\}$ \;
}
\;
\nl $\FuncSty{Refine}(\AlgVar{future},E,v,t)$ \;
\nl \Begin{
\nl    $\AlgVar{future'} \gets \emptyset$ \label{NewFuture} \;
\nl    $E' \gets E$ \;
\nl    \For {$\langle \setG,\setH,a \rangle \in future$ \label{InnerLoop}}{
\nl        $\setG' \gets \FuncSty{EdgesIncidentToVertex}(\setG,v)$ \label{NewPWithEdge} \;
\nl        $\setH' \gets \FuncSty{EdgesIncidentToVertex}(\setH,t)$ \;
\nl        \If {$\setG' \neq \emptyset$ \bf{and} $\setH' \neq \emptyset$}{
\nl          \lIf {$a=2$}{
               $\AlgVar{future'} \gets \AlgVar{future'} \cup \{\langle \setG', \setH', 1 \rangle\}$
             }
\nl          \lElse{
               $E' \gets E' \cup (g,h)$ where $g$ is the unique element of $\setG'$ and $h$ is the unique element of $\setH'$ \label{AddToE}
             }
           }
\nl        $\setG' \gets \FuncSty{EdgesNotIncidentToVertex}(\setG,v)$ \label{NewPWithoutEdge} \;
\nl        $\setH' \gets \FuncSty{EdgesNotIncidentToVertex}(\setH,t)$ \;
\nl        \If {$\setG' \neq \emptyset$ \bf{and} $\setH' \neq \emptyset$}{
              $\AlgVar{future'} \gets \AlgVar{future'} \cup \{\langle \setG', \setH', a \rangle\}$
           }
       }
\nl   $\KwSty{return}$~$(\AlgVar{future'}, E')$ \;
}
\;
\nl $\FuncSty{Search}(\AlgVar{future},M,E)$ \;
\nl \Begin{
%\nl \lIf {$\AlgVar{future} = \emptyset$ \bf{and} $|M| > |\AlgVar{incumbent}|$}
\nl \lIf {$|E| > |\AlgVar{incumbent}|$}{$\AlgVar{incumbent} \gets E$} \label{StoreIncumbent}
%\nl \lIf {$\AlgVar{future} = \emptyset$}{return}
\medskip
\nl $\AlgVar{bound} \gets |E|  + \sum_{\langle \setG,\setH,a \rangle \in \AlgVar{future}} \min(|\setG|,|\setH|)$ \label{CalcBound} \;
\nl \lIf {$\AlgVar{bound} \leq |\AlgVar{incumbent}|$}{\KwSty{return}} \label{PruneSearch}
\medskip
\nl $\langle \setG,\setH,a \rangle \gets \FuncSty{SelectLabelClass}(\AlgVar{future})$ \label{SelectClass} \;
\nl $(v,w) \gets \FuncSty{SelectEdge}(\setG)$ \label{SelectEdge} \;
\nl \For {$(t,u) \in \setH$ \label{WLoop}} {
\nl    \If {$a=2$}{  \label{aEqualsTwo}
\nl      \For(\(\triangleright\) Try both orientations) {$(t',u') \in \{(t,u), (u,t)\}$} {
\nl          $(\AlgVar{future'}, E') \gets \FuncSty{Refine}(\AlgVar{future}, E, v, t')$ \;
\nl          $(\AlgVar{future''}, E'') \gets \FuncSty{Refine}(\AlgVar{future'}, E', w, u')$ \;
\nl          $\FuncSty{Search}(\AlgVar{future''}, M\cup \{(v,t')\} \cup \{(w,u')\}, E'')$ \label{ExpandWithtu} \;
        }
       } \Else {
\nl        $(\AlgVar{future'}, E') \gets \FuncSty{Refine}(\AlgVar{future}, E, w, u)$ \;
\nl        $\FuncSty{Search}(\AlgVar{future'}, M \cup \{(w,u)\}, E')$ \label{ExpandWithu} \;
       }
  }
\nl $\setG' \gets \setG \setminus \{(v,w)\}$ \label{RemoveVW} \;
\nl $\AlgVar{future} \gets \AlgVar{future} \setminus \{\langle \setG,\setH \rangle\}$\;
\nl \lIf {$\setG' \neq \emptyset$} {$\AlgVar{future} \gets \AlgVar{future} \cup \{\langle \setG',\setH \rangle \}$}
\nl $\FuncSty{Search}(\AlgVar{future},M,E)$ \label{ExpandWithoutVW} \;
}
\;
\nl $\FuncSty{McSplitE}(\graphG,\graphH)$ \label{McSplitFun} \;
\nl \Begin{
    \nl $\KwSty{global}~\AlgVar{incumbent} \gets \emptyset$ \;
\nl $\FuncSty{Search}(\{\langle E(\graphG),E(\graphH) \rangle \},\emptyset,\emptyset)$ \label{FirstExpandCall} \;
\nl $\KwSty{return}$~$\AlgVar{incumbent}$ \;
}
\caption{Finding a maximum common subgraph.}
\label{McSplitAlg}
\end{algorithm}

The recursive procedure,
$\FuncSty{Search}$, has three parameters.  The parameter $\AlgVar{future}$ is a
list of label classes, each represented as a $\langle \setG, \setH, a\rangle$
triple.  In this triple, $\setG$ and $\setH$ are sets of edges in the two
graph.  The integer $a$ takes the value 2 if target edges may be used in either
orientation, and the value 1 if pattern edges may only be mapped to pattern
edges in the stored orientation.\footnote{In the latter case, all edges in
$\setG$ start with the same vertex $v$, all edges in $\setH$ start with the
same vertex $w$, and $v$ has already been mapped to $w$.}  The parameter $M$ is
the current mapping of vertices, and $E$ is the current mapping of edges.  On
each call to $\FuncSty{Search}$, the following invariants hold:

\begin{itemize}
    \item a pair of edges may be added to $E$ if and only if they belong to the same label class in $\AlgVar{future}$
    \item the target-graph edge in this pair may have its vertices reversed if and only if the label class has integer tag 2
    \item $((v,w),(t,u)) \in E \implies (v,t) \in M \wedge (w,u) \in M$ 
\end{itemize}

(Note that the following constraint may not be satisfied:
\[
(v,t) \in M \wedge (w,u) \in M \wedge \{v,w\} \in E(\graphG) \wedge \{t,u\} \in E(\graphH) \implies ((v,w),(t,u)) \in E.
\]

This does not affect correctness, but does imply that our model does not propagate as strongly as
the CP model with $\implies$ replaced by $\Longleftrightarrow$,
even if that model is solved using only forward checking.)

\Lineref{StoreIncumbent} stores the current edge mapping $E$ if it is large enough
to unseat the incumbent.  \Linerangeref{CalcBound}{PruneSearch} prune the
search when a calculated upper bound is not larger than the incumbent.

The remainder of the procedure performs the search.  A label class $\langle
\setG, \setH, a \rangle$ is selected from $\AlgVar{future}$ using some
heuristic (\lineref{SelectClass}); from this label class, an edge $(v,w)$ is
selected from $\setG$ (\lineref{SelectEdge}). We now iterate over all edges
$(t,u)$ in $\setH$, exploring the consequences of adding $((v,w),(t,u))$ to $E$
and adding the corresponding vertex mappings $(v,t)$ and $(w,u)$ to $M$
(\linerangeref{WLoop}{ExpandWithu}).  The $\FuncSty{Refine}$ function
creates a refined set of label-classes that results from a vertex mapping,
any also moves implied edge-edge mappings to $E$.

If $a=2$ (\lineref{aEqualsTwo}), then we have to use $\FuncSty{Refine}$ to propagate
both endpoint mappings.  Otherwise, we know that the first endpoint must already have
been mapped in an earlier recursive call, and it is sufficient to propagate only the second
endpoint mapping.  
A recursive call is made (\lineref{ExpandWithtu} or \lineref{ExpandWithu}),
on return from which we remove the edge mapping $((v,w),(t,u))$.
Having explored all possible mappings of $(v,w)$ with edges in $\setH$ we now
consider what happens if $(v,w)$ is not matched
(\linerangeref{RemoveVW}{ExpandWithoutVW}).

We start our search at the function $\FuncSty{McSplitE}$ (\lineref{McSplitFun}),
with graphs $\graphG$ and $\graphH$ as inputs.  This function returns a mapping of
maximum cardinality.  In \lineref{FirstExpandCall} the initial call is made to
$\FuncSty{Search}$; at this point we have a single label-class containing all
vertices, and the mappings $M$ and $E$ are empty.

?? Actually, we can delete any mention of the vertex mapping $M$ from the algorithm and it still works fine!

\section{A clique model using oriented edges}

\section{A clique model without oriented edges}

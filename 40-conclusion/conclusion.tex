\chapter{Conclusion}
\label{c:conclusion}

This dissertation has introduced the \McSplit\ family of algorithms ---
\McSplit\ for maximum common induced subgraph and \McSplit-SI for induced
subgraph isomorphism.  We have shown the close relationship between these
algorithms and both constraint programming and clique algorithms, and have
demonstrated experimentally that the \McSplit\ algorithms outperform the
existing state of the art on many families of benchmark instances.

We have applied \McSplit-SI to the problem of finding a small induced universal
graph --- that is, a graph that contains every member of a given set of graphs
as an induced subgraph --- which has received much theoretical attention on
asymptotic results but very little attention on the development of solvers for
small instances.  For this problem, we have developed new exact and heuristic
algorithms for the problem and used these to generate new terms that have been
included in the Online Encyclopedia of Integer Sequences.

The strength of \McSplit\ --- its compact data structures that require little
memory and enable very fast and simple filtering algorithms --- has the
disadvantage that it makes the algorithm rather inflexible.  For example, there
appears to be no obvious way to add supplemental graphs to \McSplit-SI because
they would break the invariant that two domains must be equal or disjoint.  Yet
I would argue that \McSplit\ and \McSplit-SI are far from local optima, and
that there remain many promising avenues for future development of the
algorithms.  We could explore new variable and value ordering heuristics, and
indeed work by other authors subsequent to the work presented in this thesis
(TODO cite) has shown the effectiveness of more sophisticated heuristics.  We
could try new search strategies beyond depth-first traversal of the search
tree; initial work on using solution-biased search for \McSplit\ showed
promising results, and this could extended to \McSplit-SI.

Finally, it would be interesting to look for other constraint satisfaction
problems in which many domains and constraints are equal or similar, for which
we might use a \McSplit-like data structure.

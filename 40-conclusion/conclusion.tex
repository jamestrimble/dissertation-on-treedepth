\chapter{Conclusion}
\label{c:conclusion}

This dissertation has introduced the \McSplit\ family of algorithms: \McSplit\
for maximum common induced subgraph (MCIS) and \McSplit-SI for the induced
subgraph isomorphism problem (ISIP).  We have seen the close relationship
between these algorithms and both constraint programming and clique algorithms.
We have also seen how the algorithms' data structures, combined with the
special properties of the problems, give us good upper bounds for MCIS and
generalised arc consistency on the all-different constraint in ISIP at almost
no computational cost.

The \McSplit\ algorithms outperform
the existing state of the art on many families of benchmark instances.  In the
case of MCIS, this is particularly true for unlabelled, undirected instances;
the clique encoding remains the best approach for instances with many labels on
vertices and edges.  For induced subgraph isomorphism, \McSplit-SI is the
fastest solver for many classes of graphs, including both random graphs and
some classes of structured graphs such as the new knight's grid instances
proposed by Knuth.  But
\McSplit-SI is not the best ISIP solver for every instance---for some pairs of
structured graphs from the Stanford GraphBase, for example,
supplemental graphs give the Glasgow solver an advantage.

In addition to the new solvers presented in this dissertation,
our study of algorithms for MCIS and ISIP yields the following
simple conclusions which should be applicable to a broad
range of current and future solvers.

\begin{itemize}
    \item Branch and bound is not always the best optimisation strategy
        for MCIS. We have seen that \McSplitDown, which solves a sequence
        of decision problems, can be orders of magnitude faster than
        the branch-and-bound solver \McSplit, particularly on instances
        where the maximum common induced subgraph is almost as large as
        the input graphs.
    \item It can be helpful to swap the two input graphs before calling
        an MCIS solver, particularly if the graphs differ
        in order or density.
    \item In both MCIS and ISIP, choosing the best strategy for sorting the vertices
        of each graph is a non-trivial problem, and the decision
        should take into account the properties
        of the graphs.  It can be helpful, for example, to sort in decreasing
        order of degree if the target graph is sparse and in increasing order of
        degree if the target graph is dense.
\end{itemize}

Turning from subgraphs to supergraphs,
we have applied \McSplit-SI to the problem of finding a small induced universal
graph---that is, a graph that contains every member of a given set of graphs as
an induced subgraph. This problem has received much theoretical attention,
but there is almost no prior work on the development of solvers for
small instances.  We have developed new exact and heuristic algorithms for the
problem and used these to generate new terms that have been included in the
Online Encyclopedia of Integer Sequences.

\subsection{Future Work}

The strength of \McSplit\ --- its compact data structure that requires little
memory and enables fast and simple filtering algorithms --- has the
disadvantage that it places limits on how the algorithm can be modified.  For example, there
appears to be no obvious way to add supplemental graphs to \McSplit-SI because
they would break the invariant that two domains must be equal or disjoint.  Despite
this limitation,
many promising avenues remain for future development of the
algorithms. For example:

\begin{itemize}

    \item We could explore new variable and value ordering heuristics.
        Other authors have already shown that reinforcement learning
        can be used to improve the heuristics in \McSplit\
       \citep{DBLP:conf/aaai/0001LJ020,trummer2021}.
       The present author, in preliminary work not presented here,
        has found that a version of the $\mathsf{dom/deg}$ heuristic
	\citep{DBLP:conf/cp/BessiereR96} solves more of the ISIP
	instances in \Cref{subsec:si-decision-experiment}
	than the smallest-domain-first approach that we have used.

    \item We could use new search strategies beyond depth-first traversal of
    the search tree. Initial work on using solution-biased search
    \citep{DBLP:conf/cpaior/ArchibaldDHMP019} for \McSplit\ showed promising
    results, and this could extended to \McSplit-SI.

    \item We could develop new swapping rules for \McSplit\ like those in
    \Cref{c:swapping-graphs-mcsplit}, using characteristics of the graphs other
    than order and degree.

    \item We could create a portfolio solver that chooses an appropriate
    solver for each instance from a set of solvers including \McSplit.
    This has already been done successfully for MCIS by \citep{dilkas2018};
    future work could extend this by adding the swapping versions of
    \McSplit\ from \Cref{c:swapping-graphs-mcsplit}.  There does not
    yet exist a portfolio solver for ISIP that uses \McSplit-SI.

    \item We could solve the maximum common \emph{edge} subgraph by
    using \McSplit\ to solve MCIS on the line graphs of the two input
    graphs \citep{DBLP:journals/jcamd/RaymondW02a}.  The same method
    could be used to solve subgraph monomorphism
    using \McSplit-SI.

\end{itemize}

Finally, it would be interesting to look for other constraint satisfaction
problems in which many domains and constraints are equal or similar, for which
we might use a \McSplit-like data structure as a domain store.

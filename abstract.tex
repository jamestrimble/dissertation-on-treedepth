\vspace*{1.75in}
\begin{center} {\bf Abstract}\end{center}

This dissertation introduces the \McSplit\ family of algorithms for
two closely-related NP-hard problems that involve finding a large
induced subgraph contained by each of two input graphs: the induced subgraph
isomorphism problem and the maximum common induced subgraph problem.

The \McSplit\ algorithms resemble forward-checking constrant programming
algorithms, but use problem-specific data structures that allow
multiple, identical domains to be stored without duplication.  These data structures
enable fast, simple constraint propagation algorithms and very fast calculation of
upper bounds.  Versions of these algorithms for both sparse and dense graphs are
described and implemented.
The resulting algorithms are over an order of magnitude faster
than the best existing algorithm for maximum common induced subgraph on unlabelled
graphs, and outperform the state of the art on several classes of induced subgraph
isomorphism instances.

A further advantage of the \McSplit\ data structures is that variables and
values are treated identically; this allows us to choose to branch on variables
representing vertices of either input graph with no overhead.  An extensive
set of experiments shows that such two-sided branching can be particularly
beneficial if the two input graphs have very different orders or densities.

Finally, we turn from subgraphs to supergraphs, tackling the problem of
finding a small graph that contains every member of a given family of graphs
as an induced subgraph.  Exact and heuristic techniques are developed for
this problem, in each case using a \McSplit\ algorithm as a subroutine.
These algorithms allow us to add new terms to two entries of the
On-Line Encyclopedia of Integer Sequences.
